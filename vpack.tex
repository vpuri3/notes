%
% in preamble, %
% in preamble, %
% in preamble, %
% in preamble, \include{vpack}
%
% convention
% 
% Lowercase Greek: Scalars (Reals)
% Uppercase Greek/Latin: Operator (Matrix)
% Lowercase Latin Bold: Vector Field (over some domain)
% Lowercase Latin Underlined: Discretized Vector Field
% Double Underlined: Second Order Tensor Field
%
% (u,v) . : Inner Product
% <T,u> . : Function application T(u)
% S[x'](x): Operator S acting on x', evaluated at x
% 

\usepackage{amsfonts,amssymb,amsbsy,amsmath}
\usepackage{enumitem}
\usepackage{physics}
\usepackage{siunitx}       % \SI
\usepackage{cancel}        % \cancel
\usepackage{listings}      % source code formatting

\usepackage{soul}          % \hl
\usepackage{float}         % float positioning option \begin{figure}[H]
\usepackage{graphicx}      % \includegraphics  
\usepackage{subfigure}
%\usepackage{tikz}         % stick figures
\usepackage{xcolor}
\definecolor{periwinkledark}{RGB}{102, 102, 128}
\newcommand{\vp}[1]{\textcolor{periwinkledark} {VP: #1 }}

\usepackage{hyperref} % \autoref
\renewcommand{\chapterautorefname}{Chapter}
\renewcommand{\sectionautorefname}{Section}

\newcommand{\todo}[1]{\hl{todo: #1}}
\newcommand{\vv}{vis-\`a-vis }

%================================ THEOREMS ===============================%
\usepackage{amsthm}
\theoremstyle{definition}
\newtheorem{theorem}{Theorem}[section]
\newtheorem{proposition}{Proposition}[section]
\newtheorem{remark}{Identity}[section]
\newtheorem{example}{Example}[section]
\newtheorem{corollary}{Corollary}[theorem]
\newtheorem{lemma}[theorem]{Lemma}
\newtheorem{definition}{Definition}[section]

%\renewcommand\qedsymbol{$\blacksquare$}
%================================== MATHS ==================================%
\newcommand{\eqn}[1]{ % numbered equation environment
    \begin{equation}
    \begin{aligned}
        #1
    \end{aligned}
    \end{equation}
}
\newcommand{\eqnn}[1]{ % unnumbered equation environment
    \begin{equation*}
    \begin{aligned}
        #1
    \end{aligned}
    \end{equation*}
}
\newcommand{\unit}[1]{\ensuremath{\, \mathrm{#1}}}             % units
\newcommand{\nvect}[1]{\underline{#1}}                         % discretized scalar field
\newcommand{\vect}[1]{\boldsymbol{#1}}                         % vector field
\newcommand{\uvect}[1]{\boldsymbol{\hat{#1}}}                  % unit vector (field)
\newcommand{\tensor}[1]{\underline{\underline{#1}}}            % second order tensor field
\newcommand{\der}{\,\mathrm{d}}                                % dx ==> \der x total derivative
\newcommand{\Der}{\,\mathrm{D}}                                % Du ==> \Der u material derivative
\newcommand{\ppp}[1]{\partial_{#1}}                            % partial derivative
\newcommand{\vvv}{\delta}                                      % variation
\newcommand{\del}{\nabla}                                      % gradient vector operator
\newcommand{\grad}{\vect{\del}}                                % gradient vector operator
\newcommand{\ddd}[1]{\dfrac{\mathrm{d}}{\mathrm{d} #1}}        % total derivative
\newcommand{\material}{\mathrm{D}_{t}}                         % material derivative (wrt time)
\newcommand{\defeq}{\mathrel{\mathop:}=}                       % define equal to
\newcommand{\degree}{^\circ}                                   % degree
\newcommand{\vvec}[1]{\begin{pmatrix}#1\end{pmatrix}}          % vector components (# \\ #)
\newcommand{\mat}[1]{\begin{bmatrix}#1\end{bmatrix}}           % matrix [# & # \\ # & #]
\newcommand{\transp}{\intercal}                                % matrix transpose A^\transp
\newcommand{\expo}[1]{\mathrm{e}^{#1}}                         % e = 2.71..
\newcommand{\al}{\alpha}                                       % \alpha
\newcommand{\supp}[1]{\mathrm{supp}(#1)}                       % support of a function
\newcommand{\subsubset}{\subset\subset}                        % A compact subset of B
\renewcommand{\st}{\text{ such that }}                         % such that
\newcommand{\contradiction}{\ensuremath{\Rightarrow\!\Leftarrow}} % contradiction
\newcommand{\rt}[2]{\sqrt[#1]{#2}}                             % root
\newcommand{\Re}{\text{Re}}                                    % Reynolds Number
\newcommand{\Pe}{\text{Pe}}                                    % Peclet Number
\newcommand{\opap}[1]{\langle #1 \rangle}                      % operator application <T,x>
\newcommand{\inr} [1]{( #1 )}                                  % inner product (f,g)
\newcommand{\floor}[1]{\left\lfloor #1 \right\rfloor}
\newcommand{\ceil }[1]{\left\lceil  #1 \right\rceil}
\newcommand{\N}{\mathbb{N}}                                    % Set of naturals
\newcommand{\Z}{\mathbb{Z}}                                    % Set of integers
\newcommand{\Q}{\mathbb{Q}}                                    % Set of rationals
\newcommand{\R}{\mathbb{R}}                                    % Set of reals
\renewcommand{\L}{\mathcal{L}}                                 % Space of linear operators
\newcommand{\I}{\mathcal{I}}                                   % Operator
\newcommand{\S}{\mathcal{S}}                                   % Operator
\newcommand{\D}{\mathcal{D}}                                   % Operator
\newcommand{\V}{\mathcal{V}}                                   % Operator
\renewcommand{\ul}[1]{\underline{#1}}
\renewcommand{\bar}[1]{\overline{#1}}
%
% convention
% 
% Lowercase Greek: Scalars (Reals)
% Uppercase Greek/Latin: Operator (Matrix)
% Lowercase Latin Bold: Vector Field (over some domain)
% Lowercase Latin Underlined: Discretized Vector Field
% Double Underlined: Second Order Tensor Field
%
% (u,v) . : Inner Product
% <T,u> . : Function application T(u)
% S[x'](x): Operator S acting on x', evaluated at x
% 

\usepackage{amsfonts,amssymb,amsbsy,amsmath}
\usepackage{enumitem}
\usepackage{physics}
\usepackage{siunitx}       % \SI
\usepackage{cancel}        % \cancel
\usepackage{listings}      % source code formatting

\usepackage{soul}          % \hl
\usepackage{float}         % float positioning option \begin{figure}[H]
\usepackage{graphicx}      % \includegraphics  
\usepackage{subfigure}
%\usepackage{tikz}         % stick figures
\usepackage{xcolor}
\definecolor{periwinkledark}{RGB}{102, 102, 128}
\newcommand{\vp}[1]{\textcolor{periwinkledark} {VP: #1 }}

\usepackage{hyperref} % \autoref
\renewcommand{\chapterautorefname}{Chapter}
\renewcommand{\sectionautorefname}{Section}

\newcommand{\todo}[1]{\hl{todo: #1}}
\newcommand{\vv}{vis-\`a-vis }

%================================ THEOREMS ===============================%
\usepackage{amsthm}
\theoremstyle{definition}
\newtheorem{theorem}{Theorem}[section]
\newtheorem{proposition}{Proposition}[section]
\newtheorem{remark}{Identity}[section]
\newtheorem{example}{Example}[section]
\newtheorem{corollary}{Corollary}[theorem]
\newtheorem{lemma}[theorem]{Lemma}
\newtheorem{definition}{Definition}[section]

%\renewcommand\qedsymbol{$\blacksquare$}
%================================== MATHS ==================================%
\newcommand{\eqn}[1]{ % numbered equation environment
    \begin{equation}
    \begin{aligned}
        #1
    \end{aligned}
    \end{equation}
}
\newcommand{\eqnn}[1]{ % unnumbered equation environment
    \begin{equation*}
    \begin{aligned}
        #1
    \end{aligned}
    \end{equation*}
}
\newcommand{\unit}[1]{\ensuremath{\, \mathrm{#1}}}             % units
\newcommand{\nvect}[1]{\underline{#1}}                         % discretized scalar field
\newcommand{\vect}[1]{\boldsymbol{#1}}                         % vector field
\newcommand{\uvect}[1]{\boldsymbol{\hat{#1}}}                  % unit vector (field)
\newcommand{\tensor}[1]{\underline{\underline{#1}}}            % second order tensor field
\newcommand{\der}{\,\mathrm{d}}                                % dx ==> \der x total derivative
\newcommand{\Der}{\,\mathrm{D}}                                % Du ==> \Der u material derivative
\newcommand{\ppp}[1]{\partial_{#1}}                            % partial derivative
\newcommand{\vvv}{\delta}                                      % variation
\newcommand{\del}{\nabla}                                      % gradient vector operator
\newcommand{\grad}{\vect{\del}}                                % gradient vector operator
\newcommand{\ddd}[1]{\dfrac{\mathrm{d}}{\mathrm{d} #1}}        % total derivative
\newcommand{\material}{\mathrm{D}_{t}}                         % material derivative (wrt time)
\newcommand{\defeq}{\mathrel{\mathop:}=}                       % define equal to
\newcommand{\degree}{^\circ}                                   % degree
\newcommand{\vvec}[1]{\begin{pmatrix}#1\end{pmatrix}}          % vector components (# \\ #)
\newcommand{\mat}[1]{\begin{bmatrix}#1\end{bmatrix}}           % matrix [# & # \\ # & #]
\newcommand{\transp}{\intercal}                                % matrix transpose A^\transp
\newcommand{\expo}[1]{\mathrm{e}^{#1}}                         % e = 2.71..
\newcommand{\al}{\alpha}                                       % \alpha
\newcommand{\supp}[1]{\mathrm{supp}(#1)}                       % support of a function
\newcommand{\subsubset}{\subset\subset}                        % A compact subset of B
\renewcommand{\st}{\text{ such that }}                         % such that
\newcommand{\contradiction}{\ensuremath{\Rightarrow\!\Leftarrow}} % contradiction
\newcommand{\rt}[2]{\sqrt[#1]{#2}}                             % root
\newcommand{\Re}{\text{Re}}                                    % Reynolds Number
\newcommand{\Pe}{\text{Pe}}                                    % Peclet Number
\newcommand{\opap}[1]{\langle #1 \rangle}                      % operator application <T,x>
\newcommand{\inr} [1]{( #1 )}                                  % inner product (f,g)
\newcommand{\floor}[1]{\left\lfloor #1 \right\rfloor}
\newcommand{\ceil }[1]{\left\lceil  #1 \right\rceil}
\newcommand{\N}{\mathbb{N}}                                    % Set of naturals
\newcommand{\Z}{\mathbb{Z}}                                    % Set of integers
\newcommand{\Q}{\mathbb{Q}}                                    % Set of rationals
\newcommand{\R}{\mathbb{R}}                                    % Set of reals
\renewcommand{\L}{\mathcal{L}}                                 % Space of linear operators
\newcommand{\I}{\mathcal{I}}                                   % Operator
\newcommand{\S}{\mathcal{S}}                                   % Operator
\newcommand{\D}{\mathcal{D}}                                   % Operator
\newcommand{\V}{\mathcal{V}}                                   % Operator
\renewcommand{\ul}[1]{\underline{#1}}
\renewcommand{\bar}[1]{\overline{#1}}
%
% convention
% 
% Lowercase Greek: Scalars (Reals)
% Uppercase Greek/Latin: Operator (Matrix)
% Lowercase Latin Bold: Vector Field (over some domain)
% Lowercase Latin Underlined: Discretized Vector Field
% Double Underlined: Second Order Tensor Field
%
% (u,v) . : Inner Product
% <T,u> . : Function application T(u)
% S[x'](x): Operator S acting on x', evaluated at x
% 

\usepackage{amsfonts,amssymb,amsbsy,amsmath}
\usepackage{enumitem}
\usepackage{physics}
\usepackage{siunitx}       % \SI
\usepackage{cancel}        % \cancel
\usepackage{listings}      % source code formatting

\usepackage{soul}          % \hl
\usepackage{float}         % float positioning option \begin{figure}[H]
\usepackage{graphicx}      % \includegraphics  
\usepackage{subfigure}
%\usepackage{tikz}         % stick figures
\usepackage{xcolor}
\definecolor{periwinkledark}{RGB}{102, 102, 128}
\newcommand{\vp}[1]{\textcolor{periwinkledark} {VP: #1 }}

\usepackage{hyperref} % \autoref
\renewcommand{\chapterautorefname}{Chapter}
\renewcommand{\sectionautorefname}{Section}

\newcommand{\todo}[1]{\hl{todo: #1}}
\newcommand{\vv}{vis-\`a-vis }

%================================ THEOREMS ===============================%
\usepackage{amsthm}
\theoremstyle{definition}
\newtheorem{theorem}{Theorem}[section]
\newtheorem{proposition}{Proposition}[section]
\newtheorem{remark}{Identity}[section]
\newtheorem{example}{Example}[section]
\newtheorem{corollary}{Corollary}[theorem]
\newtheorem{lemma}[theorem]{Lemma}
\newtheorem{definition}{Definition}[section]

%\renewcommand\qedsymbol{$\blacksquare$}
%================================== MATHS ==================================%
\newcommand{\eqn}[1]{ % numbered equation environment
    \begin{equation}
    \begin{aligned}
        #1
    \end{aligned}
    \end{equation}
}
\newcommand{\eqnn}[1]{ % unnumbered equation environment
    \begin{equation*}
    \begin{aligned}
        #1
    \end{aligned}
    \end{equation*}
}
\newcommand{\unit}[1]{\ensuremath{\, \mathrm{#1}}}             % units
\newcommand{\nvect}[1]{\underline{#1}}                         % discretized scalar field
\newcommand{\vect}[1]{\boldsymbol{#1}}                         % vector field
\newcommand{\uvect}[1]{\boldsymbol{\hat{#1}}}                  % unit vector (field)
\newcommand{\tensor}[1]{\underline{\underline{#1}}}            % second order tensor field
\newcommand{\der}{\,\mathrm{d}}                                % dx ==> \der x total derivative
\newcommand{\Der}{\,\mathrm{D}}                                % Du ==> \Der u material derivative
\newcommand{\ppp}[1]{\partial_{#1}}                            % partial derivative
\newcommand{\vvv}{\delta}                                      % variation
\newcommand{\del}{\nabla}                                      % gradient vector operator
\newcommand{\grad}{\vect{\del}}                                % gradient vector operator
\newcommand{\ddd}[1]{\dfrac{\mathrm{d}}{\mathrm{d} #1}}        % total derivative
\newcommand{\material}{\mathrm{D}_{t}}                         % material derivative (wrt time)
\newcommand{\defeq}{\mathrel{\mathop:}=}                       % define equal to
\newcommand{\degree}{^\circ}                                   % degree
\newcommand{\vvec}[1]{\begin{pmatrix}#1\end{pmatrix}}          % vector components (# \\ #)
\newcommand{\mat}[1]{\begin{bmatrix}#1\end{bmatrix}}           % matrix [# & # \\ # & #]
\newcommand{\transp}{\intercal}                                % matrix transpose A^\transp
\newcommand{\expo}[1]{\mathrm{e}^{#1}}                         % e = 2.71..
\newcommand{\al}{\alpha}                                       % \alpha
\newcommand{\supp}[1]{\mathrm{supp}(#1)}                       % support of a function
\newcommand{\subsubset}{\subset\subset}                        % A compact subset of B
\renewcommand{\st}{\text{ such that }}                         % such that
\newcommand{\contradiction}{\ensuremath{\Rightarrow\!\Leftarrow}} % contradiction
\newcommand{\rt}[2]{\sqrt[#1]{#2}}                             % root
\newcommand{\Re}{\text{Re}}                                    % Reynolds Number
\newcommand{\Pe}{\text{Pe}}                                    % Peclet Number
\newcommand{\opap}[1]{\langle #1 \rangle}                      % operator application <T,x>
\newcommand{\inr} [1]{( #1 )}                                  % inner product (f,g)
\newcommand{\floor}[1]{\left\lfloor #1 \right\rfloor}
\newcommand{\ceil }[1]{\left\lceil  #1 \right\rceil}
\newcommand{\N}{\mathbb{N}}                                    % Set of naturals
\newcommand{\Z}{\mathbb{Z}}                                    % Set of integers
\newcommand{\Q}{\mathbb{Q}}                                    % Set of rationals
\newcommand{\R}{\mathbb{R}}                                    % Set of reals
\renewcommand{\L}{\mathcal{L}}                                 % Space of linear operators
\newcommand{\I}{\mathcal{I}}                                   % Operator
\newcommand{\S}{\mathcal{S}}                                   % Operator
\newcommand{\D}{\mathcal{D}}                                   % Operator
\newcommand{\V}{\mathcal{V}}                                   % Operator
\renewcommand{\ul}[1]{\underline{#1}}
\renewcommand{\bar}[1]{\overline{#1}}
%
% convention
% 
% Lowercase Greek: Scalars (Reals)
% Uppercase Greek/Latin: Operator (Matrix)
% Lowercase Latin Bold: Vector Field (over some domain)
% Lowercase Latin Underlined: Discretized Vector Field
% Double Underlined: Second Order Tensor Field
%
% (u,v) . : Inner Product
% <T,u> . : Function application T(u)
% S[x'](x): Operator S acting on x', evaluated at x
% 

\usepackage{amsfonts,amssymb,amsbsy,amsmath}
\usepackage{enumitem}
\usepackage{physics}
\usepackage{siunitx}       % \SI
\usepackage{cancel}        % \cancel
\usepackage{listings}      % source code formatting

\usepackage{soul}          % \hl
\usepackage{float}         % float positioning option \begin{figure}[H]
\usepackage{graphicx}      % \includegraphics  
\usepackage{subfigure}
%\usepackage{tikz}         % stick figures
\usepackage{xcolor}
\definecolor{periwinkledark}{RGB}{102, 102, 128}
\newcommand{\vp}[1]{\textcolor{periwinkledark} {VP: #1 }}

\usepackage{hyperref} % \autoref
\renewcommand{\chapterautorefname}{Chapter}
\renewcommand{\sectionautorefname}{Section}

\newcommand{\todo}[1]{\hl{todo: #1}}
\newcommand{\vv}{vis-\`a-vis }

%================================ THEOREMS ===============================%
\usepackage{amsthm}
\theoremstyle{definition}
\newtheorem{theorem}{Theorem}[section]
\newtheorem{proposition}{Proposition}[section]
\newtheorem{remark}{Identity}[section]
\newtheorem{example}{Example}[section]
\newtheorem{corollary}{Corollary}[theorem]
\newtheorem{lemma}[theorem]{Lemma}
\newtheorem{definition}{Definition}[section]

%\renewcommand\qedsymbol{$\blacksquare$}
%================================== MATHS ==================================%
\newcommand{\eqn}[1]{ % numbered equation environment
    \begin{equation}
    \begin{aligned}
        #1
    \end{aligned}
    \end{equation}
}
\newcommand{\eqnn}[1]{ % unnumbered equation environment
    \begin{equation*}
    \begin{aligned}
        #1
    \end{aligned}
    \end{equation*}
}
\newcommand{\unit}[1]{\ensuremath{\, \mathrm{#1}}}             % units
\newcommand{\nvect}[1]{\underline{#1}}                         % discretized scalar field
\newcommand{\vect}[1]{\boldsymbol{#1}}                         % vector field
\newcommand{\uvect}[1]{\boldsymbol{\hat{#1}}}                  % unit vector (field)
\newcommand{\tensor}[1]{\underline{\underline{#1}}}            % second order tensor field
\newcommand{\der}{\,\mathrm{d}}                                % dx ==> \der x total derivative
\newcommand{\Der}{\,\mathrm{D}}                                % Du ==> \Der u material derivative
\newcommand{\ppp}[1]{\partial_{#1}}                            % partial derivative
\newcommand{\vvv}{\delta}                                      % variation
\newcommand{\del}{\nabla}                                      % gradient vector operator
\newcommand{\grad}{\vect{\del}}                                % gradient vector operator
\newcommand{\ddd}[1]{\dfrac{\mathrm{d}}{\mathrm{d} #1}}        % total derivative
\newcommand{\material}{\mathrm{D}_{t}}                         % material derivative (wrt time)
\newcommand{\defeq}{\mathrel{\mathop:}=}                       % define equal to
\newcommand{\degree}{^\circ}                                   % degree
\newcommand{\vvec}[1]{\begin{pmatrix}#1\end{pmatrix}}          % vector components (# \\ #)
\newcommand{\mat}[1]{\begin{bmatrix}#1\end{bmatrix}}           % matrix [# & # \\ # & #]
\newcommand{\transp}{\intercal}                                % matrix transpose A^\transp
\newcommand{\expo}[1]{\mathrm{e}^{#1}}                         % e = 2.71..
\newcommand{\al}{\alpha}                                       % \alpha
\newcommand{\supp}[1]{\mathrm{supp}(#1)}                       % support of a function
\newcommand{\subsubset}{\subset\subset}                        % A compact subset of B
\renewcommand{\st}{\text{ such that }}                         % such that
\newcommand{\contradiction}{\ensuremath{\Rightarrow\!\Leftarrow}} % contradiction
\newcommand{\rt}[2]{\sqrt[#1]{#2}}                             % root
\newcommand{\Re}{\text{Re}}                                    % Reynolds Number
\newcommand{\Pe}{\text{Pe}}                                    % Peclet Number
\newcommand{\opap}[1]{\langle #1 \rangle}                      % operator application <T,x>
\newcommand{\inr} [1]{( #1 )}                                  % inner product (f,g)
\newcommand{\floor}[1]{\left\lfloor #1 \right\rfloor}
\newcommand{\ceil }[1]{\left\lceil  #1 \right\rceil}
\newcommand{\N}{\mathbb{N}}                                    % Set of naturals
\newcommand{\Z}{\mathbb{Z}}                                    % Set of integers
\newcommand{\Q}{\mathbb{Q}}                                    % Set of rationals
\newcommand{\R}{\mathbb{R}}                                    % Set of reals
\renewcommand{\L}{\mathcal{L}}                                 % Space of linear operators
\newcommand{\I}{\mathcal{I}}                                   % Operator
\newcommand{\S}{\mathcal{S}}                                   % Operator
\newcommand{\D}{\mathcal{D}}                                   % Operator
\newcommand{\V}{\mathcal{V}}                                   % Operator
\renewcommand{\ul}[1]{\underline{#1}}
\renewcommand{\bar}[1]{\overline{#1}}