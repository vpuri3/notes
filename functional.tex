\chapter{Functional Analysis}

\section{Multi-Index Notation}
A multi-index is a $d$-tuple of non-negative integers $\alpha=(\alpha_1,\dotsc,\alpha_d).$ The degree of a multi-index $\alpha$ is $\abs{\alpha}=\sum_{i=1}^d\alpha_i$. Consider a point in $d$-dimensional real Euclidean space $x=(x_1,\dotsc,x_d)\in\mathbb{R}^d.$ We denote by $x^\alpha$ the monomial $x_1^{\alpha_1}\cdots x_d^{\alpha_d}.$ If $\Der_{x_i}=\ppp{x_i},$ denotes the partial differential operator with respect to variable $x_i,$ then 
$\Der^\alpha$ denotes a differential operator of order $\abs{\alpha}.$
\eqn{
    \Der^\alpha = \Der_{x_1}^{\alpha_1}\cdots \Der_{x_d}^{\alpha_d} = \ppp{x_1}^{\alpha_1}\cdots\ppp{x_d}^{\alpha_d}
}
For two multi-indices $\beta,\,\alpha,$ we say $\beta\leq\alpha$ if $\beta_i\leq\alpha_i$ for $1\leq i\leq d.$ In this case, $\alpha\pm\beta$ are also multi-indices with elements $\alpha_i\pm\beta_i$ and order $\abs{\alpha\pm\beta}=\abs{\alpha}\pm\abs{\beta}.$ We also denote $\alpha!=(\alpha_1!,\dotsc,\alpha_d!)$ and if $\beta\leq\alpha,$
\eqn{
    \vvec{\alpha\\\beta} = \frac{\alpha!}{\beta!(\alpha-\beta)!} =
    \big( \vvec{\alpha_1\\\beta_1},\dotsc,
    \vvec{\alpha_d\\\beta_d}\big)
}

\begin{theorem}
    \textit{Multinomial Theorem}
    \eqn{
        (x_1+\dotsb+x_d)^k = \sum_{\abs{\al}=k}\vvec{k\\\al}x^\al
    }
\end{theorem}
\begin{theorem}[Leibniz Formula]
\eqn{
    \Der^\alpha(uv) = \sum_{\beta\leq\al}\vvec{\alpha\\\beta}\Der^\beta u \Der^{\alpha-\beta}v
}    
\end{theorem}
\begin{proof}
We prove the Leibniz formula by induction on $\abs{\alpha}.$ It is trivial to show that the equality holds for $\abs{\alpha}=0,1.$ Assuming the equality holds for for some $\alpha\neq0,$ we prove that the equality holds for $\alpha+e_i$ where is a multi-index with $1$ in the $i^\mathrm{th}$ position $\forall i\in\{1,\cdots,d\}$ and zeros elsewhere.
\eqn{
    \Der^{\al+e_i}(uv)
    &=
    \Der^{e_i}\bigg(\sum_{\beta\leq\al}\vvec{\alpha\\\beta}
    \Der^{\beta}u\Der^{\al-\beta}v\bigg)
    =
    \sum_{\beta\leq\al}\vvec{\alpha\\\beta}
    \Der^{e_i}\big(\Der^{\beta}u\Der^{\al-\beta}v\big)
    \\&=
    \sum_{\beta\leq\al}\vvec{\alpha\\\beta}\Der^{\beta+e_i}u\Der^{\al-\beta}v +     \vvec{\alpha\\\beta}\Der^{\beta}u\Der^{\al-\beta+e_i}v
}

    
\end{proof}

\begin{theorem}
    \textit{(Taylor's Formula)} For sufficiently smooth functions $f:\mathbb{R}^d\rightarrow\mathbb{R},$
    \eqn{
        f(x) = \sum_{\abs{\al}\leq k}\frac{1}{\al!}\Der^{\al}f(x_0)(x-x_0)^\al + \mathcal{O}(\abs{x}^{k+1})
    }
\end{theorem}

%%%%%%%%%%%%%%%%%%%%%%%%%%%%%%%%%%%%%%%%%%%%%%%%%%%%%%%%%%%%%%%%%%%%%%%%%%%%%%%%%%%%%%%%%%%%%
\section{Spaces and Norms}
For the purposes of this text, a topological vector space (TVS) is a vector space for which the vector operations of addition and scalar multiplication are continuous. That is for $x,y\in X,\,c\in\mathbb{C}$ where $X$ is some TVS, the maps $(x,y)\rightarrow x+y$ and $(c,x)\rightarrow cx$ are continuous.

A \textbf{norm} on a vector space $X$ is a real-valued functional $\norm{\cdot}:X\rightarrow\mathbb{R}$ such that for $x,y\in X,\,c\in\mathbb{C},$
\begin{itemize}
    \item $\norm{x}\geq0$ with equality holding iff $x=0.$
    \item $\norm{cx}=\abs{c}\norm{x}$
    \item $\norm{x+y}\leq\norm{x}+\norm{y}$
\end{itemize}
A TVS is normable if its topology coincides with that induced by some norm. Two norms $\norm{\cdot}_1 ,\,\norm{\cdot}_2$ are equivalent if for all $x\in X,$ there exists some $c\in\mathbb{R}$ such that
\eqn{
    c\norm{x}_1 \leq \norm{x}_2 \leq \frac{1}{c}\norm{x}_1
}
If $X$ is a normed space and all Cauchy sequences converge to a limit in $X$ with respect to the norm, then $X$ is called a \textbf{Banach space}. A vector space $X$ is called \textbf{pre-Hilbert} if there exists a functional called the \textbf{scalar product} $(\cdot,\cdot):X\times X\rightarrow\mathbb{R}$ such that $\forall u,v\in X,$
\begin{itemize}
    \item $(u,u)\geq 0.\,(u,u)=0\iff u=0$
    \item $(u,v)=(v,u)$
    \item $(u_1+u_2,v)=(u_1,v)+(u_2,v),\,u_1,u_2\in X$
    \item $(\lambda u,v)=\lambda(u,v),\,\lambda\in\mathbb{R}.$
\end{itemize}
The scalar product induces a norm $\norm{u}=\sqrt{(u,u)}.$ Another functional defined on a vector space $X$ is the \textbf{inner product} $(\cdot,\cdot):X\cross X\rightarrow \mathbb{C}$ such that for $x,y,z\in X,\, a,b\in\mathbb{C},$
\begin{itemize}
    \item $(x,y) = \overline{(y,x)}$
    \item $(ax+by,z) = a(x,z) + b(y,z)$
    \item $(x,x)=0\iff x=0$
\end{itemize}
If $X$ is a Banach space under a norm that is induced by an inner product, then $X$ is called a \textbf{Hilbert space}. Given an inner-product, a norm on $X$ can be specified as follows: for $x\in X,$
\eqn{
    \norm{x} = (x,x)^{1/2}
}
We consider maps (or operators) between vector spaces $X$ and $Y$, $L:X\rightarrow Y$. A map is linear if $\forall x_1,x_2\in X,\forall \al\in\mathbb{R},f(x_1+\al x_2)=f(x_1)+\al f(x_2).$ For example, differentiation is a linear map from $C^n$ to $C^{n-1}$, and integration from $L^1_\mathrm{loc}$ to $\mathbb{R}$. Every linear map $L:\mathbb{R}^n\rightarrow\mathbb{R}^m$ has an associated $m\cross n$ matrix $A$ written as follows (where $\nvect{e_j}$ are unit vectors in the $j^\mathrm{th}$ cardinal direction):
\eqn{
    \nvect{y} = A\nvect{x} &= \mat{L(\nvect{e_1}) & \dots & L(\nvect{e_n})}_{m\cross n} \cdot \vvec{x_1\\\vdots\\x_n}_{n\cross 1} = \sum_{j=1}^n x_j L(\nvect{e_j}) = L\left( \sum_{j=1}^n x_j\nvect{e_j} \right) = L(\nvect{x})
}
A linear operator is said to be bounded if $\exists C>0\st \norm{Lx}_X\leq C\norm{x}_X$
\begin{theorem}
    For linear operators, the following are equivalent.
    \begin{enumerate}
        \item $L$ is bounded
        \item $L$ is continuous at one point
        \item $L$ is continuous for all inputs
    \end{enumerate}
\end{theorem}
We denote with $\mathcal{L}(X,Y)$ the vector space of  bounded linear operators from $X$ to $Y$. We define the following induced norm on $\mathcal{L}(X,Y)$
\eqn{
    \norm{L} &= \sup_{\norm{x}_X=1}\norm{Lx}_Y
}
\begin{theorem}
    If $X$ and $Y$ are normed vector spaces and $Y$ is complete, then $\mathcal{L}(X,Y)$ is complete.
\end{theorem}
A map from $X$ to its field, $f:X\rightarrow \mathbb{R}$ is called a \textbf{functional}. The set of linear functionals on $X$ is called the \textbf{dual} of $X$ and is denoted by $X',$ which is also a vector space under pointwise addition and scalar multiplication. We represent the application of a functional on an element as follows:
\eqn{
    f(x) = \component{f,x}
}
The operator norm on dual space $X'$ is
\eqn{
    \norm{x'}_{X'} = \sup_{\norm{x}_X=1} \abs{x'(x)}
}
\begin{theorem}
    \textit{(Riesz Representation Theorem)} For a Hilbert space $X,$ a linear functional $x'$ on $X$ belongs in $X'$ if and only if there exists an $y\in X$ such that for every $x\in X,$
    \eqn{
        x'(x) = (x,y)_X
    }
\end{theorem}

We consider an open subset $\Omega\in\mathbb{R}^d$ as domain in $d$-dimensional Euclidean space. For some $S\subset\Omega,$ we denote by $\overline{S}$ the closure of $S.$ We write $S\subset\subset\Omega\subset\mathbb{R}^d$ if $\overline{S}\subset\Omega$ and $\overline{S}$ is compact. If $u$ is a function defined on $\Omega,$ we define the support of $u$ as
\eqn{
    \supp{u} =\overline{\{ x\in\Omega| u(x)\neq 0\}}
}
We say $A\subsubset B$ if and only if $\overline{A}$ is a compact subset of $B.$ If, for some function $u,$ $\supp{u}\subsubset\Omega,$ we say that $u$ has \textbf{compact support} in $\Omega.$ A measurable function $u$ on $\Omega$ is said to be essentially bounded on $\Omega$ if there exists a constant $K$ such that $\abs{u(x)}\leq K$ a.e. on $\Omega.$ The greatest lower bound of such constants $K$ is called the essential supremum of $\abs{u}$ on $\Omega$ and is denoted by $\mathrm{ess}\sup_{x\in\Omega}\abs{u(x)}.$ We denote by $L^\infty(\Omega)$ the vector space consisting of all equivalent classes of functions $u$ (that are equal almost everywhere) which are essentially bounded on $\Omega.$
\eqn{
    \norm{u}_\infty = \mathrm{ess}\sup_{x\in\Omega}\abs{u(x)}
}
We define the following family of vector spaces. For some integer $m,$ the vector space $C^m(\Omega)$ consists of all functions defined on $\Omega$ which, along with their partial derivatives up to order $m$ are continuous.
\eqn{
    C^m(\Omega) &= \{ \phi:\Omega\rightarrow \mathbb{R}: \forall\al\text{ with }\abs{\alpha}\leq m,\,\Der^\alpha\phi\hspace{4} \text{is continuous}\}
    \\
    C^\infty(\Omega)&=\cap_{m=0}^\infty C^m(\Omega)
}
We abbreviate $C^0(\Omega)$ to $C(\Omega).$ The subspace $C_0^m(\Omega)$ consists of all functions in $C^m(\Omega)$ that have compact support in $\Omega.$ We wish to define equivalent spaces for $\overline{\Omega},$ $C^m(\overline{\Omega}).$ However, since $\Omega$ is open, continuous functions on $\Omega$ need not be bounded. Therefore, not all functions in $C^m(\Omega)$ can be continuously extended to $\overline{\Omega}.$ For example, the function $f(x)=1/x$ is continuous on the open interval $(0,1),$ but discontinuous on $x=0.$ Functions that are uniformly continuous and bounded on $\Omega$ can be uniquely and continuously extended to $\overline{\Omega}.$ We define $C^\m(\overline{\Omega})$ as follows:
\eqn{
    C^m(\overline{\Omega}) = \{\phi\in C^m(\Omega):\forall\al,\,\abs{\al}\leq m,\, \Der^\al\phi\text{ bounded and uniformly continuous on }\Omega\}
}
A function $f:\Omega\rightarrow\mathbb{R}$ is said to be \textbf{H\"older continuous} for exponent $\gamma$ ($0<\gamma\leq1$) if
\eqn{
    \exists K\geq0\,\forall x,y\in\Omega,x\neq y,\frac{\abs{f(x)-f(y)}}{\norm{x-y}}\leq K
}
We define $C^{m,\gamma}(\overline{\Omega})$ to be the subspace of $C^m(\overline{\Omega})$ consisting of functions for which, for $\abs{\alpha}=m,$ $\Der^\alpha\phi$ satisfies the H\"older condition. We define the following norms and seminorms:
\begin{align}
    \norm{u}_{C^0(\overline{\Omega})} &= \sup_{x\in\Omega}\abs{\Der^\alpha u}
    &
    \abs{u}_{C^{0,\gamma}(\overline{\Omega})} &= \sup_{\substack{x,y\in\Omega\\x\neq y}}\frac{\abs{u(x)-u(y)}}{\norm{x-y}_2}
    \\
    \norm{u}_{C^{m}(\overline{\Omega})} &= \max_{0\leq\abs{\al}\leq m}\norm{u}_{C^0(\overline{\Omega})}
    &
    \norm{u}_{C^{m,\gamma}(\overline{\Omega})} &= \norm{u}_{C^m(\overline{\Omega})}+\max_{\abs{\al}=m}\abs{\Der^\al u}_{C^{0,\gamma}(\overline{\Omega})}
\end{align}
\begin{theorem}
    $C^{m,\gamma}(\Omega)$ is complete for nonnegative integer $m$ and $0<\gamma\leq1$ with respect to the norm $\norm{\cdot}_{C^{m,\gamma}(\overline{\Omega})}.$
\end{theorem}
\begin{proof}
    Observe that $\forall x\in\Omega,\,\{u_n(x)\}\subset\mathbb{R}$ form a Cauchy sequence. By completeness of $\mathbb{R},\forall x, u_x(x)\rightarrow u(x).$ It is left to prove that $\lim_{n\rightarrow\infty}\norm{u-u_n}_{C^{m,\gamma}(\Omega)}=0.$
\end{proof}
Let $\Omega$ be a domain in $\mathbb{R}^d$ and let $p\in\mathbb{R}^+.$ We denote by $L^p(\Omega)$ the class of all measurable functions $u:\Omega\rightarrow\mathbb{R}$ such that
\eqn{
    \int_{\Omega}\abs{u(x)}^p\der x < \infty
}
The elements of $L^p(\Omega)$ are equivalent classes of functions that are equal almost everywhere on $\Omega$ that satisfy the above inequality. We show that $L^p(\Omega)$ is a vector space. We say $u=0$ in $L^p(\Omega)$ if $u$ is zero almost everywhere on $\Omega$ (i.e. $u(x)=0\,\mathrm{a.e.}$ in $\Omega$). The restriction to $\Omega$ of any piecewise continuous, compactly supported function on $\mathbb{R}^d$ would belong to an equivalent class of functions in $L^p(\Omega).$ For $u,v\in L^p(\Omega),\,x\in\Omega,\,c\in\mathbb{C},$
\eqn{
    \abs{u(x)+v(x)}^p \leq (\abs{u(x)}+\abs{v(x)})^p \leq 2^p(\abs{u(x)}^p+\abs{v(x)}^p)
}
Hence $u+v\in L^p(\Omega).$ Clearly $cu(x)\in L^p(\Omega).$ Therefore $L^p(\Omega)$ is a vector space. We claim that the function $\norm{\cdot}_p$ defined below is a norm on $L^p(\Omega)$ for $1\leq p<\infty.$ $\norm{\cdot}_p$ is not a norm for $0<p<1.$
\eqn{
    \norm{u}_p = \bigg\{ \int_{\Omega}\abs{u(x)}^p\der x \bigg\}^{1/p}
}
For $u\in L^p(\Omega),$ it is clear that $\norm{u}_p\geq0$ with equality holding if and only if $u=0$ in $L^p(\Omega).$ Moreover, $\norm{cu}_p=\abs{c}\norm{u}_p.$ It remains to be shown that $\norm{\cdot}_p$ satisfies the triangle inequality, in $L^p(\Omega),$ which is known as \textit{Minkowski's inequality}. The condition certainly holds for $p=1,$ since
\eqn{
    \int_\Omega\abs{u(x)+v(x)}\der x \leq 
    \int_\Omega\abs{u(x)}\der x +
    \int_\Omega\abs{v(x)}\der x
}
For $p>1,$ we denote by $p'$ the number $\frac{p}{p-1}$ so that $p'>1$ and
\eqn{
    \frac{1}{p} + \frac{1}{p'} = 1
}
We call $p'$ the exponential conjugate to $p.$

\begin{theorem}
    \textit{(H\"older's inequality)} For $p>1,\,u\in L^p(\Omega)\,v\in L^{p'}(\Omega),$ then $uv\in L^1(\Omega)$ and
    \eqn{
        \int_{\Omega}\abs{u(x)v(x)}\der x = \norm{uv}_1 \leq \norm{u}_p\norm{v}_{p'}
    }
\end{theorem}
\begin{proof}
    If either $\norm{u}_{p}=0$ or $\norm{v}_{p'}=0$ then $u(x)v(x)=0$ almost everywhere and the inequality is satisfied. Otherwise, consider the function $f:\mathbb{R}^+\rightarrow\mathbb{R},\,f(t)=t^p/p+1/p'-t.$ The only critical point of the function is at $f(1)=0,$ which is a minimum. Hence for all $t,$
    \eqn{
        t \leq \frac{t^p}{p} + \frac{1}{p'}
    }
    For $a,b\geq0,$ we substitute $t=ab^{-p'/p},$
    \eqn{
        ab^{-p'/p} &\leq \frac{a^pb^{-p'}}{p} + \frac{1}{p'}
        \\
        ab^{-p'/p+p'} &\leq \frac{a^pb^{-p'+p'}}{p} + \frac{b^{p'}}{p'}
        \\
        ab &\leq \frac{a^p}{p} + \frac{b^{p'}}{p'}
        \\
    }
    with equality occurring if and only if $a^p=b^{p'}.$ Substitute $a=\frac{\abs{u(x)}}{\norm{u}_{p}},\,b=\frac{\abs{v(x)}}{\norm{v}_{p'}}$ and integrate over $\Omega.$
    \eqn{
    \int_{\Omega}\frac{\abs{u(x)}\abs{v(x)}}{\norm{u}_{p}\norm{v}_{p'}}\der x
    &\leq
    \frac{1}{p}\int_{\Omega} \frac{\abs{u(x)}^p}   {\norm{u}_{p}^p}    \der x + \frac{1}{p'}\int_{\Omega}\frac{\abs{v(x)}^{p'}}{\norm{v}_{p'}^{p'}}\der x
    =
    \frac{1}{p} + \frac{1}{p'}
    \\
    \int_\Omega\abs{u(x)v(x)}\der x &\leq \int_\Omega\abs{u(x)}\abs{v(x)}\der x
    \leq \norm{u}_{p}\norm{v}_{p'}
    }
    Hence, $uv\in L^1(\Omega).$
\end{proof}
From H\"older's inequality, we get that each $v\in L^{p'}(\Omega)$ is associated with an element in the dual of $L^p(\Omega)$, $Fu &= \int_\Omega u v \der x$. The case $p=p'=2$ is special because every element in $L^2(\Omega)$ can be identified with an element in its dual space $L^2(\Omega)'.$
\begin{theorem}
    \textit{(Minkowski's Inequality)} Triangle Inequality for $p\geq1,\,u,v\in L^p(\Omega).$
    \eqn{
        \norm{u+v}_{p} \leq \norm{u}_p + \norm{v}_p
    }
\end{theorem}
\begin{proof}
    Since the case with $p=1$ is trivial, consider $p>1.$ For $u,v\in L^p(\Omega),\,u+v\in L^p(\Omega).$
    \eqn{
        \int_\Omega \abs{u(x)+v(x)}^{p'(p-1)} \der x
        =\int_\Omega \abs{u(x)+v(x)}^p \der x < \infty
    }
    Hence, $(u+v)^{p-1}\in L^{p'}(\Omega).$
    \eqn{
        \norm{u+v}_{p}^p =
        \int_\Omega \abs{u(x)+v(x)}^p \der x
        &\leq
        \int_\Omega \abs{u(x)+v(x)}^{p-1}(\abs{u(x)}+\abs{v(x)}) \der x
        \\
        \int_\Omega \abs{u(x)+v(x)}^p \der x
        &\leq
        \bigg\{\int_\Omega\abs{u(x)+v(x)}^{p'(p-1)}     \der x\bigg\}^{1/p'}
        \bigg\{\int_\Omega\abs{\abs{u(x)}+\abs{v(x)}}^p \der x\bigg\}^{1/p}
        \\
        \bigg\{\int_\Omega\abs{u(x)+v(x)}^p \der x\bigg\}^{1-1/p'}
        &\leq
        \bigg\{\int_\Omega\abs{\abs{u(x)}+\abs{v(x)}}^p \der x\bigg\}^{1/p}
        \\
        \bigg\{\int_\Omega\abs{u(x)+v(x)}^p \der x\bigg\}^{1/p}
        &\leq
        \bigg\{\int_\Omega\abs{u(x)}^p \der x\bigg\}^{1/p} +
        \bigg\{\int_\Omega\abs{v(x)}^p \der x\bigg\}^{1/p}
        \\
        \norm{u+v}_{p} &\leq \norm{u}_p + \norm{v}_p
    }
    Therefore, $\norm{\cdot}_p$ is a norm on $L^p(\Omega).$
\end{proof}
\begin{theorem}[Hahn-Banach Theorem]
    Let $X$ be a Banach space and $X_0$ be a subspace of $X$. $\forall x_0'\in X_0',\exists\, x'\in X'\st$ $x_0$ and $x'$ agree on $X_0$, and $\norm{x_0'}_{X_0'}=\norm{x'}_{X}$.
\end{theorem}
\begin{theorem}
    If $X$ is a normed vector space, then $X'$, along with the induced operator norm, is a banach space (irrespective of whether $X$ is complete).
\end{theorem}
\begin{theorem}[Generalised Cauchy Identity]
    $\abs{(x',x)} \leq \norm{x}_X\norm{x'}_{X'}$
\end{theorem}
%%%%%%%%%%%%%%%%%%%%%%%%%%%%%%%%%%%%%%%%%%%%%%%%%%%%%%%%%%%%%%%%%%%%%%%%%%%%%%%%
\section{Adjoint Operator}
We explain how a linear transformation between two normed vector spaces, $L\in\mathcal{L}(X,Y)$, is naturally associated with another transformation from $Y'$ to $X'$ by the \textbf{adjoint operator}, $L^*:Y'\rightarrow X'$, as follows: $\forall y'\in Y,\exists$ unique $x'\in X'\st \forall x\in X, y'\circ L(x) = x'(x) \implies \forall y'\in Y',\forall x\in X$,
\eqn{
    L^*y'(x) &= y'\circ L(x) \in X'\\
    y' &\to Ly'\\
    \component{y',L(x)} &= \component{L^*y',x}
}
The adjoint operator is also bounded since $\abs{y'\circ L (x)}\leq \norm{y'}_{Y'}\norm{L(x)}_Y\leq \norm{y'}_{Y'}\norm{L}\norm{x}_{X}$.

We show that for a map $L:\mathbb{R}^n\rightarrow\mathbb{R}^m$ with associated matrix $A$, the matrix of the adjoint is $A^\transp$. The action of the adjoint operator is one that for every $\nvect{b}\in\mathbb{R}^m$ produces a unique $\nvect{c}\in\mathbb{R}^n\st$
\eqn{
    &\nvect{c}^\transp\nvect{x} = \nvect{b}^\transp A \nvect{x}, \hspace{1em}\forall \nvect{x}\in\mathbb{R}^n\\
    \implies &\nvect{x}^\transp\left(\nvect{c} - A^\transp \nvect{b} \right) = 0 \\
    \implies &\nvect{c} = A^\transp \nvect{b}
}
Most differential operators will be defined on a subset of $L^2(\Omega)$. For linear operators on space $L^2(\Omega)$, we have
\eqn{
    \component{v',L(u)} &= \component{L^*v',u},\hspace{1em}&\forall u\in L^2(\Omega),v\in L^2(\Omega)'\\
    \implies (v,Lu) &= (L^*v,u),&\forall u,v\in L^2(\Omega)
}
\begin{definition}[Adjoint Operator]
    For differential operator $L$, the adjoint is the differential operator $L^*$ that satisfies
    \eqn{
        \left(Lu,v\right) &= \left(u,L^*v\right)
    }
    for sufficiently smooth $u,v$ with compact support in $\Omega$.
\end{definition}
\begin{theorem}
    For $L:X\rightarrow Y$, $\norm{L^*} = \norm{L}$
\end{theorem}
\begin{proof}
    \eqn{
        \abs{y'\circ L(x)} &\leq \norm{y'}\norm{L}\norm{x},\hspace{1em}&\forall x\in X,\forall y'\in Y'\\
        \frac{\abs{y'\circ L(x)}}{\norm{y'}\norm{x}} &\leq \norm{L},&\forall x\neq 0, \forall y'\neq 0\\
        \norm{L^*} &\leq \norm{L}, &\text{by taking supremum over $x, y'$}
    }
    Now we prove the converse
    \eqn{
        \abs{y'\circ L(x)} &\leq \norm{L^*}\norm{y'}\norm{x},\hspace{1em}&\forall x,y'\\
        \frac{\abs{y'\left(L(x)\right)}}{\norm{y'}\norm{x}}&\leq \norm{L^*},&\forall x\neq 0,\forall y'\neq 0\\
        \norm{L} &\leq \norm{L^*}, &\text{by taking supremum over $x, y'$}
    }
\end{proof}
Adjoints are useful in understanding problems like the following: given normed vector spaces $X,Y$ and mapping $\mathcal{L}(X,Y)$ and $b\in Y$, find $x\in X\st$
\eqn{
    Lx = b
}
Consider the $m\cross n$ system $Ax=b$. We enlarge the system by adding the adjoint system $c = A^\transp y$ as follows:
\eqn{
    \mat{0 & A \\ A^\transp & 0} \vvec{y\\x} &= \vvec{b\\c}
}
\todo{}
Consider the problem
\eqn{
    \left\{
    \begin{split}
        &Lu = f,&x\in\Omega\\
        &\text{suitable b.c, i.c,}\hspace{1em}&x\in\partial\Omega
    \end{split}
    \right.
}
The Green's function $\phi(x,y)$ satisfies
\eqn{
    \left\{
    \begin{split}
        &L^*\phi(x,y) = \delta(x-y),&x\in\Omega\\
        &\text{adjoint b.c, i.c,}\hspace{1em}&x\in\partial\Omega
    \end{split}
    \right.
}
We obtain the following representation formula:
\eqn{
    u(y) &= (u(x),\delta(x-y)) = (u(x),L^*\phi(x,y))\\
         &= (Lu(x),\phi(x,y)) = (f(x),\phi(x,y))\\
         &= \int_\Omega f(x)\phi(x,y)\der x
}
We enlarge the system as follows:
\eqn{
    \left\{
    \begin{split}
        &Lu = f,&x\in\Omega\\
        &\text{suitable b.c, i.c,}\hspace{1em}&x\in\partial\Omega\\
        &L^*\phi(x,y) = \delta(x-y),&x\in\Omega\\
        &\text{adjoint b.c, i.c,}\hspace{1em}&x\in\partial\Omega\\
    \end{split}
    \right.
}
%%%%%%%%%%%%%%%%%%%%%%%%%%%%%%%%%%%%%%%%%%%%%%%%%%%%%%%%%%%%%%%%%%%%%%%%%%%%%%%%
\section{Some Imbeddings}
We say that the normed space $X$ is imbedded in the normed space $Y$ and write $X\rightarrow Y$ if:
\begin{itemize}
    \item $X$ is a vector subspace of $Y.$
    \item the identity operator, $I$ defined on X into Y is continuous. Since $I$ is linear, it is equivalent to the existence of a constant $M$ such that, for $x\in X,$
    \eqn{
        \norm{I(x)}_Y \leq M\norm{x}_X
    }
\end{itemize}
For nonnegative integer $m$ and $0<\nu\leq\lambda\leq 1,$ we prove the following imbeddings:
\begin{itemize}
    \item $C^{m+1}(\overline{\Omega})\rightarrow C^m(\overline{\Omega})$ \\
    It is clear that $C^{m+1}(\overline{\Omega})\subset C^m(\overline{\Omega})$ and $\norm{\phi}_{C^m(\overline{\Omega})}\leq\norm{\phi}_{C^{m+1}(\overline{\Omega})}$

    \item $C^{m,\lambda}(\overline{\Omega})\rightarrow C^m(\overline{\Omega})$ \\
    It is clear that $C^{m,\lambda}(\overline{\Omega})\subset C^m(\overline{\Omega})$ and $\norm{\phi}_{C^m(\overline{\Omega})}\leq\norm{\phi}_{C^{m,\lambda}(\overline{\Omega})}$

    \item For $0<\nu<\lambda\leq 1,$ $C^{m,\lambda}(\overline{\Omega})\rightarrow C^{m,\nu}(\overline{\Omega})$ \\
    Consider $\phi\in C^{m,\lambda}(\overline{\Omega}).$ There exists $K$ such that for all $\alpha$ with $\abs{\alpha}\leq m,$ for all $x,y\in\Omega,\,x\neq y$
    \eqn{
        \frac{\abs{D^{\alpha}\phi(x)-D^{\alpha}\phi(y)}}{\abs{x-y}^\lambda} &< K \\
        \frac{\abs{D^{\alpha}\phi(x)-D^{\alpha}\phi(y)}}{\abs{x-y}^\nu} &< K\abs{x-y}^{\lambda-\nu} \\
    }
    The term $\abs{x-y}^{\lambda-\nu}$ is bounded for bounded $\Omega.$ Hence, $C^{m,\lambda}(\overline{\Omega})\subset C^{m,\nu}(\overline{\Omega}).$ Next, we compare the norms of $C^{m,\lambda}(\overline{\Omega})$ and $C^{m,\nu}(\overline{\Omega}).$ For some $\phi\in C^{m,\lambda}(\overline{\Omega}),$ we note that
    \eqn{
    \sup_{\substack{x,y\in\Omega\\\abs{x-y}<1}} \frac{\abs{\Der^\alpha\phi(x)-\Der^\alpha\phi(y)}}{\abs{x-y}^{\nu}}
    &\leq 
    \sup_{\substack{x,y\in\Omega\\\abs{x-y}<1}} \frac{\abs{\Der^\alpha\phi(x)-\Der^\alpha\phi(y)}}{\abs{x-y}^{\lambda}}
    \\
    \sup_{\substack{x,y\in\Omega\\\abs{x-y}\geq1}} \frac{\abs{\Der^\alpha\phi(x)-\Der^\alpha\phi(y)}}{\abs{x-y}^{\nu}}
    &\leq 
    2\sup_{\substack{x\in\Omega}}\abs{\Der^\alpha\phi(x)}
    }
    Hence, $\norm{\phi}_{C^{m,\nu}(\overline{\Omega})} \leq 3\norm{\phi}_{C^{m,\lambda}(\overline{\Omega})}$
\end{itemize}



H\"older's inequality \hl{clearly} extends to cover this case with $p=\infty,\,p'=1.$ We now prove an imbedding theorem on $L^p(\Omega).$
\begin{theorem}
    Suppose $\mathrm{vol}\Omega=\int_\Omega\der x,$ and $1\leq p\leq q\leq\infty.$ If $u\in L^q(\Omega),$ then $u\in L^p(\Omega)$ and
    \eqn{
        \norm{u}_p\leq(\mathrm{vol}\Omega)^{1/p-1/q}\norm{u}_q
    }
    Hence, $L^q(\Omega)\rightarrow L^p(\Omega).$ If $u\in L^\infty(\Omega),$ then
    \eqn{
        \lim_{p\rightarrow\infty}\norm{u}_p = \norm{u}_\infty
    }
    Finally, if $u\in L^p(\Omega)$ for $1\leq p <\infty,$ and if there is a constant $K$ such that for all $p,$ $\norm{u}_p\leq K,$ then $u\in L^\infty(\Omega)$ and $\norm{u}_\infty\leq K.$
\end{theorem}
\begin{proof}
    Consider $u\in L^q(\Omega).$ If $p=q,$ the imbedding theorem is trivial. For $1\leq p < q \leq \infty,$
    Hence $u^{p}\in L^{q/p}(\Omega).$ Then, H\"older's inequality gives us
    \eqn{
        \int_\Omega\abs{u(x)}^p\der x
        &\leq
        \bigg\{\int_\Omega\abs{u(x)}^{p(q/p)}\der x\bigg\}^{p/q}
        \bigg\{\int_\Omega 1                 \der x\bigg\}^{1-p/q}
        \\
        \norm{u}_{p} &\leq \mathrm{vol}(\Omega)^{1/p-1/q}\norm{u}_{q}
    }
    Hence, for $1\leq p \leq q\leq\infty,\,L^q(\Omega)\rightarrow L^p(\Omega).$ If $u\in L^\infty(\Omega),$ then
    \eqn{
        \lim_{p\rightarrow\infty}\norm{u}_p &\leq \norm{u}_\infty
    }
    On the other hand, $\forall\varepsilon>0\,\exists A\subset\Omega$ of nonzero measure $\mu(A)$ such that $\forall x\in A,$
    \eqn{
        \abs{u(x)}&\geq \norm{u}_\infty - \varepsilon
        \\
        \int_\Omega\abs{u(x)}^p\der x &\geq \int_A\abs{u(x)}^p\der x \geq \mu(A)(\norm{u}_\infty-\varepsilon)^p
        \\
        \norm{u}_p &\geq (\mu(A))^{1/p}(\norm{u}_\infty-\varepsilon)
        \\
        \lim_{p\rightarrow\infty} \norm{u}_p &\geq \norm{u}_\infty
    }
    Therefore,
    \eqn{
        \lim_{p\rightarrow\infty} \norm{u}_p = \norm{u}_\infty
    }
    Finally, suppose $u\in L^p(\Omega)$ for $1\leq p< \infty$ and $\exists K$ such that for all such $p,\, \norm{u}_p\leq K.$ Suppose, \textit{ad absurdum}, $u$ is essentially bounded. Then $\exists K_1>K,\,A\subset\Omega$ with $\mu(A)>0$ such that $\forall x\in A,\,\abs{u(x)}\geq K_1.$ Applying the same argument as before, we get
    \eqn{
        \lim_{p\rightarrow\infty}\norm{u}_p \geq K_1
    }
    which is a contradiction.
\end{proof}
\begin{theorem}
    If $\Omega$ is measurable, then $L^p(\Omega)$ is a Banach space for $1\leq p\leq\infty.$
\end{theorem}
\begin{theorem}
    For $f\in L^p(\Omega)\cap L^q(\Omega),\,1\leq p\leq q\leq \infty,$ then $\forall r\in[p,q],f\in L^r(\Omega)$ and $\norm{f}_r\leq\norm{f}_p^\al\norm{f}_q^{1-\al}$ where $\dfrac{\al}{p}+\dfrac{1-\al}{q}=\dfrac{1}{r}.$
\end{theorem}
\begin{proof}
    The proof is trivial for $r=p$ or $r=q.$ If $p<r<q,$ then $\dfrac{1}{p}>\dfrac{1}{r}>\dfrac{1}{q}.$ Then, there exists $\al$ such that
    \eqn{
        \al\dfrac{1}{p}+(1-\al)\dfrac{1}{q}=\dfrac{1}{r}
    }
    Then, for any $m,n\geq1$ with $\dfrac{1}{m}+\dfrac{1}{n}=1,$ we have
    \eqn{
        \int_\Omega\abs{f(x)}^r\der x =\int_\Omega\abs{f(x)}^{\al r}\abs{f(x)}^{(1-\al)r}
        \leq \bigg\{\int_\Omega\abs{f(x)}^{r\al m}\der x\bigg\}^{1/m}
             \bigg\{\int_\Omega\abs{f(x)}^{r(1-\al)n}\der x\bigg\}^{1/n}
    }
    Choose $m=\dfrac{p}{r\al}$ and $m=\dfrac{q}{r(1-\al)}$ if $\al\neq1.$ If $\al=1,$ choose $n=\infty.$
    \eqn{
        \int_\Omega\abs{f(x)}^r\der x &\leq \norm{f}_p^{\al}\norm{f}_q^{1-\al}\leq\infty
    }
    Therefore $f\in L^r(\Omega).$
\end{proof}

\todo{We now consider the normed dual of $L^p(\Omega).$}
%%%%%%%%%%%%%%%%%%%%%%%%%%%%%%%%%%%%%%%%%%%%%%%%%%%%%%%%%%%%%%%%%%%%%%%%%%%%%%%%%%%%%%%%%%%%%
\section{Measure Theory}
The \textbf{measure} of a set refers to its size in some measure space. We say a set $S\subset\mathbb{R}^d$ has \textbf{measure zero} ($\mu(S)=0$) if $\forall\varepsilon>0,$ $S$ can be covered by open balls of total volume less than $\varepsilon.$ $\mathbb{Q}\subset\mathbb{R},$ $C^1$ curves in $\mathbb{R}^2$, $C^1$ surfaces in $\mathbb{R}^3$ are examples of sets with measure zero. A countable union of sets of measure zero has measure zero. We say that a condition holds \textbf{almost everywhere (a.e.)} if the points where it does not hold form a set of zero measure. For example, the characteristic function for rationals on the set of real numbers is equal to zero almost everywhere.

\begin{definition}[Measurable function]
    A function is measurable if it coincides a.e. with the limit of a sequence of piecewise continuous functions which is convergent almost everywhere.
\end{definition}

We define the characteristic function of a set $A\subset \mathbb{R}^d$ as follows:
\eqn{
    \chi_{A}(x) = \big\{
    \begin{array}{cc}
        1 & x\in A \\
        0 & x\notin A
    \end{array}
}

We say that a set is measurable if its characteristic function is measurable. $\mathbb{Q}$ is measurable since its characteristic function coincides with zero almost everywhere. A countable union of measurable sets is also measurable.

\begin{definition}[Piecewise Continuous]
    $g:\mathbb{R}^d\rightarrow \mathbb{R}$ is piecewise continuous if there exist disjoint, open and connected domains $\{D_i\}_{i\in I\subset\mathbb{N}}$ with piecewise $C^1$ boundary such that any sphere can be covered by finitely many $\overline{D_i}.$ Further, $g$ is continuous on each $D_i$ and can be continuously extended to the boundary of $D_i.$
\end{definition}

Consider a measurable function $f:\mathbb{R}^d\rightarrow\mathbb{R},\,f(x)\geq0.$ One can construct a nondecreasing sequence of piecewise continuous functions $\{g_k\}$ with compact support, convergent almost everywhere to $f.$ We say that $f$ is \textbf{Lebesgue integrable} if the sequence of Riemann integrals
\eqn{
    \int_{\mathbb{R}^d}g_k(x)\der x
}
has an upper bound (hence a limit since $g_k$ is a nondecreasing). One can show that $\int_{\mathbb{R}^d}g_k(x)\der x$ does not depend on the choice of sequence $g_k.$
\eqn{
    \int_{\mathbb{R}^d}f(x)\der x = \lim_{k\rightarrow\infty}\int_{\mathbb{R}^d}g_k(x)\der x
}
For example, $\chi_{\mathbb{Q}}$ is not Riemann integrable since it discontinuous everywhere, but is Lebesgue integrable since it is equal to zero a.e. We define the Lebesgue integral of a measurable function $f$ over a measurable subset $\Omega$ of $\mathbb{R}^d$ as follows:
\eqn{
    \int_\Omega f(x)\der x = \int_{\mathbb{R}^d}f(x)\chi_{\Omega}(x)\der x
}
We say a measurable function $u:\mathbb{R}^d\rightarrow\Omega$ is \textbf{summable} if $\int_{\mathbb{R}^d}\abs{u}\der x < \infty$ and define the set of summable functions
\eqn{
    L^1(\mathbb{R}^d)=\{u:\mathbb{R}^d\rightarrow\mathbb{R}:\int_{\mathbb{R}^d}\abs{u(x)}\der x <\infty \}
}
A measurable function is \textbf{locally integrable} on $\Omega\subset\mathbb{R}^d$ if it is integrable on any compact $K\subset\Omega.$ The set of locally integrable functions on $\Omega$ is denoted $L^1_\mathrm{loc}(\Omega).$
\eqn{
    L^1_\text{loc}(\Omega) = \{u:\Omega\rightarrow\mathbb{R}:\forall K\subset\Omega, \int_{K}u(x)\der x<\infty\}
}
Any piecewise continuous function is locally integrable in $\mathbb{R}^d.$ An important fact is that summable functions are ``approximately continuous" at every point.
\begin{theorem}[Lebesgue's Differentiation Theorem]
    Let $u\in L^1_\mathrm{loc}(\mathbb{R}^d).$ Then for a.e. point in $\mathbb{R}^d,$
    \eqn{
        \lim_{r\rightarrow0}\int_{B_{x_0}(r)}u(x)\der x = u(x_0)
    }
\end{theorem}
%%%%%%%%%%%%%%%%%%%%%%%%%%%%%%%%%%%%%%%%%%%%%%%%%%%%%%%%%%%%%%%%%%%%%%%%%%%%%%%%%%%%%%%%%%%%%
%%%%%%%%%%%%%%%%%%%%%%%%%%%%%%%%%%%%%%%%%%%%%%%%%%%%%%%%%%%%%%%%%%%%%%%%%%%%%%%%%%%%%%%%%%%%%
\section{Distributions}
We restate the divergence theorem: for $u\in C^1(\overline{\Omega})$ in scalar form (this can be derived from the vector form by considering a vector field with scalar $u$ being the $i^\text{th}$ component, and all other components being zero),
\eqn{
    \int_{\partial\Omega}u n_i\der S = \int_{\Omega}\ppp{x_i}u\der x
}
where $n_i$ is the $i^\text{th}$ component of the unit normal vector defined on $\partial\Omega.$ We derive the formula for integration by parts: for $u,v\in C^1(\Omega),$
\eqn{
    \int_{\partial\Omega}uvn_i\der S  &=\int_{\Omega}\ppp{x_i}(uv)\der x = \int_{\Omega}u\ppp{x_i}v+v\ppp{x_i}v \der x
    \\
    \int_{\Omega}u\ppp{x_i}v\der x &= \int_{\partial\Omega}uvn_i\der S - \int_{\Omega}v\ppp{x_i}v\der x
}

The space $C_0^\infty(\Omega)$ contains infinitely smooth functions with compact support on $\Omega.$ We call functions belonging to $C_0^\infty(\Omega)$ \textbf{test functions}. A key feature of test functions is that their extension to $\partial\Omega$ is the zero function as they are continuous and compactly supported in a subset of $\Omega.$ Let $u,\,v\in L^1_{\mathrm{loc}}(\Omega)$ and $\al$ be a multi-index. We say that $v$ is the $\al^\mathrm{th}$ \textbf{weak derivative} of $u$ if
\eqn{
    \forall\phi\in C^\infty_0(\Omega),\,\int_\Omega u\Der^\al\phi\der x = (-1)^\abs{\al}\int_\Omega v\phi\der x
}
It is straightforward to show that the weak derivative of $u,$ if it exists, is uniquely defined up to a set of measure zero. Further, if $u\in C^\abs{\al}(\Omega),$ its weak derivative is equal to its classical derivative.

A \textbf{mollifier}, $J(x),$ is a nonnegative, real-valued function belonging to $C_0^\infty(\mathbb{R}^d)$ used to create smooth functions approximating non-smooth functions via convolutions. Mollifiers satisfy the condition that $\int_{\mathbb{R}^d}J(x)\der x=1.$ We define the standard mollifier
\eqn{
    J(x) = 
    \begin{cases}
        C\exp{\frac{-1}{1-\abs{x}^2}} &\abs{x}<1
        \\
        0 &\abs{x}\geq1
    \end{cases}
}
where $C$ is chosen such that the integral of $J$ over $\mathbb{R}^d$ is equal to $1.$ For $\varepsilon>0,$ we define
\eqn{
    J_\varepsilon(x)=\varepsilon^{-d}J(x/\varepsilon)
}
which is compactly supported in the ball of radius $\varepsilon$ around the origin. If $u\in L^1_\mathrm{loc}(\mathbb{R}^d),$ we define the mollification of $u$ as the convolution
\eqn{
    u_\varepsilon(x)=J_\varepsilon*u(x)=\int_{\mathbb{R}^d}J_\varepsilon(x-y)u(y)\der y
}
We define $\Omega_\varepsilon=\{x\in\Omega:\mathrm{dist}(x,\partial\Omega)<\varepsilon\}.$
\begin{theorem}[Properties of mollifiers]
    Let $u:\mathbb{R}^d\rightarrow\mathbb{R}$ and vanishes identically outside $\Omega.$
    \begin{enumerate}
        \item If $u\in L^1_\mathrm{loc}(\overline{\Omega}),\,u_\varepsilon\in C_0^\infty(\mathbb{R}^d)$
        \item If also $u$ has compact support in $\Omega,$ then $u_\varepsilon\in C_0^\infty(\Omega)$ provided $\varepsilon<\mathrm{dist}(\supp{u},\partial\Omega)$
        \item $u_\varepsilon\rightarrow u$ as $\varepsilon\rightarrow0.$
        \item if $G\subsubset\Omega$ and $u\in C(\Omega),$ then $u_\varepsilon\rightarrow u$ uniformly on $G.$
        \item If $u\in C(\overline{\Omega}),$ then $u_\varepsilon\rightarrow u$ uniformly on $\Omega.$
    \end{enumerate}
\end{theorem}
\begin{proof}
\eqn{
    \Der^\al(J_\varepsilon*u)(x)&=\int_{\mathbb{R}^d}(\Der_x^\al J_\varepsilon(y-x))u(y)\der y
}
$J_\varepsilon(y-x)$ is infinitely differentiable, and vanishes outside $B_{x}(\varepsilon)$ for every $x$ for every multi-index $\al.$ If $\supp{u}\subsubset\Omega,$ and $\varepsilon<\mathrm{dist} (\supp{u}, \partial\Omega),$ then $\forall x\notin\Omega,\,u_\varepsilon(x)=0.$ Therefore, $u_\varepsilon$ has compact support in $\Omega.$
From the Lebesgue differentiation theorem,
\eqn{
    \lim_{r\rightarrow0}\int_{\Omega}\abs{u(y)-u(x)}\der y = 0
}
a.e. on $\Omega.$ Fix such a point $x$ on $\Omega.$ Then,
\eqn{
    \abs{u_\varepsilon(x)-u(x)} &= \abs{\int_\Omega J_\varepsilon(y-x)u(y) \mathrm{vol}(\Omega)u(x)\der y}
}
\end{proof}


%%%%%%%%%%%%%%%%%%%%%%%%%%%%%%%%%%%%%%%%%%%%%%%%%%%%%%%%%%%%%%%%%%%%%%%%%%%%%%%%
\section{Sobolev Intro}
We introduce Sobolev spaces of integer order and establish some of their basic properties. These are vector subspaces of various spaces $L^p(\Omega).$ We define the functional $\norm{\cdot}_{m,p}$ where $m$ is a nonnegative integer and $1\leq p\leq \infty$ as follows:
\eqn{
    \norm{u}_{m,p} &= \bigg\{ \sum_{0\leq\abs{\al}\leq m}\norm{\Der^\al u}_p^p \bigg\}^{1/p}
    \\
    \norm{u}_{m,\infty} &= \max_{0\leq\abs{\al}\leq m}\norm{\Der^\al u}_\infty
}
for any function for which the right side makes sense. The above functional defines a norm on any vector space of functions where the right side takes finite values provided functions are identified in the space if they are equal almost everywhere in $\Omega.$ We consider three such spaces corresponding to any given values of $m$ and $p.$
\begin{itemize}
    \item $H^{m,p}(\Omega)\equiv$ the completion of $\{u\in C^m(\Omega):\norm{u}_{m,p}<\infty\}$
    \item $W^{m,p}(\Omega)\equiv\{u\in L^p(\Omega):\Der^\al u\in L^p(\Omega)\,\mathrm{for}\,0\leq\abs{\al}\leq m\}$
    \item $W_0^{m,p}(\Omega)\equiv$ the closure of $C_0^\infty(\Omega)$ in $W^{m,p}(\Omega).$
\end{itemize}
Any compactly supported function in $C^\infty(\Omega)$ belongs to $W^{m,p}(\Omega)$ for any $m.$ If $\Omega$ is bounded, then $C^{m}(\Omega)\subset W^{m,p}(\Omega)$ for any $1\leq p\leq\infty.$
\begin{theorem}
    The space $W^{m,p}(\Omega)$ is a Banach space with respect to the norm $\norm{\cdot}_{m,p}$ where $m\in\mathbb{N},\,1\leq p\leq\infty.$2
\end{theorem}
\begin{proof}
    \hl{todo}
    Consider the sequence $\{u_n\}_n\subset W^{m,p}(\Omega)$ such that $\forall\varepsilon>0\exists N\,\mathrm{s.t.}\,\forall m,n\geq N,\,\norm{u_m-u_n}_{m,p}<\varepsilon.$
\end{proof}

A set $\Omega$ is said to have the \textbf{segment property} if $\forall x_0\in\partial\Omega,$ there exists an open neighbourhood of $x_0$ called $\Omega_{x_0}$ and a nonzero direction $y_{x_0}\in\mathbb{R}^d$ such that $\forall t\in[0,1],\,\overline{\Omega}\cap\Omega_{x_0}+ty_{x_0}\subset\Omega.$ Sets with the segment property are only present on one side of their boundary. Balls, polytopes and open sets with $C^1$ boundary have the segment property. However, not all sets with piecewise $C^1$ boundary have the segment property.
\begin{theorem}
    Let $\Omega\subset\mathbb{R}^d$  be an open set with a bounded, piecewise $C^1$ boundary. If $\Omega$ has the segment property, then for $1\leq p<\infty,$ there exists the trace operator $T:W^{1,p}(\Omega)\rightarrow L^p(\partial\Omega).$ $T$ is linear and continuous and $Tf=f|_{\partial\Omega}$
\end{theorem}
\begin{proof}
    First we define $T$ on $C_0^\infty(\mathbb{R}^d)$ and then extend it by density to $W^{1,p}(\Omega).$ For $f\in C^\infty_0(\mathbb{R}^d),\,Tf=f|_{\partial\Omega}.$ Clearly, $T$ is linear. It remains to be shown that $T$ is continuous with respect to the norm in $W^{1,p}(\Omega)$ and $L^p(\Omega),$ i.e. there exists constant $K$ such that $\forall f\in C^\infty_0(\mathbb{R}^d),$
    \eqn{
        \norm{Tf}_{L^p(\partial\Omega)} \leq K\norm{f}_{W^{1,p}(\Omega)}
    }
\end{proof}


