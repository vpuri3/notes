\chapter{Fluid Mechanics}

%%%%%%%%%%%%%%%%%%%%%%%%%%%%%%%%%%%%%%%%%%%%%%%%%%%%%%%%%%%%%%%%%%%%%%%%%%%%%%%%
\section{Fluids}
A fluid is a material that deforms continuously under action of a shear stress, however small. This contrasts with the behaviour of a solid which deforms only when stress values are in certain regimes. Solids are \textbf{elastic}: internal stresses in solids resist absolute deformation \vvis some original state; fluids are \textbf{viscous}: internal stresses in fluids resist the time rate of deformation. A corollary is that a fluid in static equilibrium supports only its weight and forces acting normal to its boundary. A material is said to be fluid if it seems to ``flow" in the timescale of observation $t_\mathrm{obs}$. That is, if the material is able to relax to a natural state in time less than $t_\mathrm{obs}.$ The relaxation time of a fluid is denoted $\lambda_\text{relax}$. The ratio of relaxation time to observation time is called Deborah Number. This dimensionless quantity is named after the prophet Deborah who, in the Book of Judges, proclaimed ``The mountains flowed before the lord." For large enough $t_\text{obs},$ even mountains will behave like fluids. A material is fluid if $\text{De}<<1.$
\eqn{
    \text{De} = \frac{\lambda_\text{relax}}{t_\text{obs}}
}
We tend to neglect the discrete, molecular nature of matter and treat the fluid to be made of a ``continuum". Macroscopic properties such as density and velocity are taken to be well defined for infinitesimal volume elements --small in comparison to system lengthscale but larger in comparison to molecular lengthscales. Fluid properties can vary continuously from one volume element to another and are average values of the molecular properties.

To derive the equations governing viscous fluid flow, we have to consider the time rate of change of quantities/properties in fluid control volumes. A control volume is an arbitrarily defined volume with a closed bounding surface. A material volume is a control volume that contains the same particles of matter at all times. A particular material volume may be defined by the closed bounding surface that envelops its material particles at a certain time. Hence, the velocity of the surface at every point is equal to the flow velocity at that point. The term ``fluid element" is synonymous with a material volume.

We consider two perspectives to view motion in a continuum: the \textbf{Lagrangian} perspective expresses position, velocity and other state variables in terms of material points travelling with a fluid; the \textbf{Eulerian} perspective expresses fluid properties with respect to some reference coordinate system. The time evolution of the position of the material point $\vect{X}$ can be tracked on an Eulerian coordinate system with a vector $\vect{x}(\vect{X},t).$ Since only one material point can be located at an Eulerian coordinate at a given time, there exists an inverse relation $\vect{X}(\vect{x},t)$ mapping material points $\vect{X}$ to the Eulerian coordinate position they occupy at time $t.$  To illustrate the difference, consider velocity. The velocity of a material point $\vect{X}(\vect{x_0},t)$ is given by $\eval{\p_{t}\vect{X}(\vect{x},t)}_{\vect{x_0}} = \vect{v}(\vect{X}\left(\vect{x_0},t),t\right)$

\begin {table}[H]
\centering  \caption{Lagrangian and Eulerian Points of View}
\label{tbl:bdf}
\begin{tabular}{|c | c | c |}
\hline
 & Eulerian & Langrangian\\\hline
Coordinate & $\vect{x}(\vect{X},t)$ & $\vect{X}(\vect{x},t)$ \\
\hline
Input &  &    \\
      & $t$ (time) & $t$ (time)\\\hline
Output &  &    \\\hline
Velocity & $\p_t\vect{x}(\vect{X},t)$ & $\p_t\vect{X}(\vect{x},t)$ \\
\hline
\end{tabular}
\end{table}

Let $\vect{F}$ be some property of a fluid at some material point $\vect{X}\in\Omega$ at time $t>0$. $\vect{F}$ depends on $\vect{X},$ i.e. which material point one chooses and time $t$ as material points may interact with other points and lose/gain $\vect{F}$ over time. Hence the property $\vect{F}$ following a material point varies with the point's Eulerian coordinate and time, i.e. $\vect{F}=\vect{F}(\vect{x},t).$ The rate of change of $\vect{F}$ following the material point is
\eqn{
    \ddd{t}\vect{F} &= \p_{t}\vect{F} + \sum_{i=1}^n\p_{x_i}\vect{F}\partial_t x_i
    \\
    \material\vect{F} &\defeq\p_{t}\vect{F} + (\vect{v}\cdot\grad)\vect{F}
}
Another way to think of the time rate of change of property $\vect{F}$ as one follows a material point is the following: $\p_{t}\vect{F}$ is the time rate of change of $\vect{F}$ at fixed position $\vect{x},$ and $\p_{x_i}\vect{F}\partial_t x_i$ is the rate of change of $\vect{F}$ as one would travel in the direction $i$ multiplied by how fast the material point is travelling in said direction $i.$

Consider two points, $\vect{x_0},\,\vect{x_0}+\vect{h}$, in a flow at time $t=0$. We call the matrix $J$ the Jacobian of $\vect{X}$ (Lagrangian coordinate) with respect to $\vect{x}$ (Eulerian coordinate).
\eqn{
    J(x_0,t) = \left[J_i_j\right] =
    \lim_{\abs{h}\to0}\frac{X_i(x_0_j+h_j,t) - X_i(x_0_j,t)}{\abs{h}} = \eval{\p_{x_j}X_i(\vect{x},t)}_{\vect{x}=\vect{x_0}}
}
An infinitesimal line element $\d V$ at time $t=0$ is stretched to $\det(J)\d V$ at time $t$. Consider the material derivative of the Jacobian. \todo{}
\eqn{
    \p_{t}\det(J) =& \p_{t}\det\left(\p_{x_j}X_i\right)
    = \sum_{i,j}\frac{\partial\det(J)}{\partial J_{ij}}\p_{t}\frac{\partial X_i}{\partial x_j}
    = \sum_{i,j}\frac{\partial\det(J)}{\partial J_{ij}}\frac{\partial v_i}{\partial x_j}\\
    =&\sum_{i,j}\frac{\partial\det(J)}{\partial J_{ij}}
        \left(\sum_k\frac{\partial v_i}{\partial X_k}\frac{\partial X_k}{\partial x_j}\right)
    = \sum_{i,k}\frac{\partial v_i}{\partial X_k}
        \left(\sum_j\frac{\partial\det(J)}{\partial J_{ij}}\frac{\partial X_k}{\partial x_j}\right)
}

%%%%%%%%%%%%%%%%%%%%%%%%%%%%%%%%%%%%%%%%%%%%%%%%%%%%%%%%%%%%%%%%%%%%%%%%%%%%%%%%
\todo{flow visualization - streamlines, streaklines, pathlines}
%%%%%%%%%%%%%%%%%%%%%%%%%%%%%%%%%%%%%%%%%%%%%%%%%%%%%%%%%%%%%%%%%%%%%%%%%%%%%%%%
\section{Stress}
Stress is defined as a force across a ``small" boundary ($\in\mathbb{R}^2$) per unit area of that boundary, for all orientations of the boundary. Stress is defined at a point with respect to a surface on which it would act. Consequently, stress depends on the orientation of the surface on which it acts. Hence, $\vect{\tau} = \vect{\tau}(\vect{x},t,\vect{n})$ where $\vect{n}$ is the outward pointing normal to the surface. In the limit $\delta A\rightarrow 0,$ stress at a point is independent of the magnitude of the area.
\begin{equation}
\begin{split}
    \vect{\tau} &= \lim_{\delta A\rightarrow 0} \dfrac{\delta \vect{F}}{\delta A}
\end{split}
\end{equation}
We probe the dependence of stress on the normal the surface it acts on. We consider a cases that reveals stress at a point acting on two sides of a surface to be equal in magnitude and opposite in direction.
Consider an infinitesimal disk-shaped fluid element with area $\delta A$ and height $\delta h.$ Without loss of generality, label the normal on one face $\vect{n}$ and the other $-\uvect{n}.$ Let $x_0$ be the middle of the disk and let $\vect{g}$ be acceleration due to some applied body force. The equation of motion for the fluid element is:
\eqn{
    \delta A\delta h\material\vect{\rho v} &= \vect{\tau}(\vect{x}_0+0.5\delta h\uvect{n},t,\uvect{n})\delta A + \vect{\tau}(\vect{x}_0-0.5\delta h\uvect{n},t,-\uvect{n})\delta A + \rho\delta A\delta h\vect{g}
    \\
    \delta h\material\vect{\rho v} &= \vect{\tau}(\vect{x}_0+0.5\delta h\uvect{n},t,\uvect{n}) + \vect{\tau}(\vect{x}_0-0.5\delta h\uvect{n},t,-\uvect{n}) + \rho\delta h\vect{g}
}
In the limit $\delta h\rightarrow 0,$ the rate of change of momentum of the disk falls off, and we get
\eqn{
    &\vect{\tau}(\vect{x}_0,t,-\uvect{n}) + \vect{\tau}(\vect{x}_0,t,\uvect{n}) = 0 \\
    &\vect{\tau}(\vect{x}_0,t,-\uvect{n}) = -\vect{\tau}(\vect{x}_0,t,\uvect{n})
}
Let stress at a point $x_0$ at time $t_0$ be a function of the normal to the surface it acts on i.e $\vect{\tau}=\vect{\tau}(\uvect{n}).$  Let $\sigma_{ij}$ be the component of stress in the $\uvect{j}$ direction on a surface with normal $\uvect{i}.$ Similarly, we have
\eqn{
    \vect{\tau}(\uvect{i}) &= \sigma_{ii}\uvect{i} + \sigma_{ji}\uvect{j} + \sigma_{ki}\uvect{k} \\
    \vect{\tau}(\uvect{j}) &= \sigma_{ij}\uvect{i} + \sigma_{jj}\uvect{j} + \sigma_{kj}\uvect{k} \\
    \vect{\tau}(\uvect{k}) &= \sigma_{ik}\uvect{i} + \sigma_{jk}\uvect{j} + \sigma_{kk}\uvect{k} \\
}
where each $\sigma_{**}$ is a scalar field varying in space and time. We now show that stress on a surface with an arbitrary normal can be represented as linear combinations of $\vect{\tau}(\uvect{i}),\, \vect{\tau}(\uvect{j}),\, \vect{\tau}(\uvect{k}).$ Consider a fluid element in the shape of an infinitesimal tetrahedral with vertices $\vect{x_0},\,\vect{x_0}+\delta x\uvect{i},\,\vect{x_0}+\delta y\uvect{j},\,\vect{x_0}+\delta z\uvect{k}$. This tetrahedral has a faces parallel to the $x\-y,\,x\-z,\,y\-z$ planes with areas $\delta A_z,\,\delta A_y,\,\delta A_x,$ and normals $-\uvect{k},\,-\uvect{j},\,-\uvect{i}$ respectively. Let the fourth face have area $\delta A$ and some arbitrary normal $\uvect{n}=n_x\uvect{i}+n_y\uvect{j}+n_z\uvect{k}.$ Let $\theta_x,\,\theta_y,\,\theta_z$ be the angle between $\uvect{n}$ and the coordinate axes. The areas are related by the following expressions:
\eqn{
    \delta A_x &= \cos\theta_x = n_x\delta A \\
    \delta A_y &= \cos\theta_y = n_y\delta A \\
    \delta A_z &= \cos\theta_z = n_z\delta A \\
}
We now apply Newton's laws to the fluid element in the limit $\delta x,\,\delta y,\,\delta z\rightarrow0.$ Since the volume of the fluid element, $\delta V$ falls much faster than any of the surface areas, the mass time acceleration and body forces go to zero faster than the surface forces.
\eqn{
\delta V   &\propto \delta x \delta y \delta z \\
\delta A   &\propto \delta z \sqrt{\delta x^2+\delta y^2} \\
\delta A_x &\propto \delta y \delta z \\
\delta A_y &\propto \delta x \delta z \\
\delta A_z &\propto \delta x \delta y \\
}
Say $\delta x,\,\delta y,\,\delta z$ go to zero as $\frac{1}{N}.$ Then, for bounded constants $c,\,c_v,$
\eqn{
    0 &=
    \delta A\vect{\tau}(\uvect{n})    +
    \delta A_x\vect{\tau}(-\uvect{i}) + 
    \delta A_y\vect{\tau}(-\uvect{i}) +
    \delta A_z\vect{\tau}(-\uvect{k}) 
    \\
    0 &=
    \delta A\vect{\tau}(\uvect{n})     +
    n_x\delta A\vect{\tau}(-\uvect{i}) + 
    n_y\delta A\vect{\tau}(-\uvect{i}) +
    n_z\delta A\vect{\tau}(-\uvect{k})
    \\    
    0 &= \lim_{N\rightarrow\infty}
    \dfrac{c  \vect{\tau}(\uvect{n})}{N^2}  +
    \dfrac{cn_x\vect{\tau}(-\uvect{i})}{N^2} +
    \dfrac{cn_y\vect{\tau}(-\uvect{j})}{N^2} +
    \dfrac{cn_z\vect{\tau}(-\uvect{k})}{N^2} +
    \dfrac{c_v}{N^3}(\rho\vect{g}-\material{\rho\vect{v}})
    \\
    0 &= \lim_{N\rightarrow\infty}
    c  \vect{\tau}(\uvect{n})  +
    cn_x\vect{\tau}(-\uvect{i}) +
    cn_y\vect{\tau}(-\uvect{j}) +
    cn_z\vect{\tau}(-\uvect{k}) +
    \dfrac{c_v}{N}(\rho\vect{g}-\material{\rho\vect{v}})
    \\
    0 &=
       \vect{\tau}(\uvect{n})  +
    n_x\vect{\tau}(-\uvect{i}) +
    n_y\vect{\tau}(-\uvect{j}) +
    n_z\vect{\tau}(-\uvect{k})
    \\
    \vect{\tau}(\uvect{n})  &=
    n_x\vect{\tau}(\uvect{i}) +
    n_y\vect{\tau}(\uvect{j}) +
    n_z\vect{\tau}(\uvect{k})
}
Hence, forces due to surface stresses must balance each other in the limit of the tetrahedral shrinking to a point. Expressing the above expression in matrix notation,
\eqn{
    \vect{\tau}(\uvect{n}) &=
                \mat{\vect{\tau}(\uvect{i}) &
                    \vect{\tau}(\uvect{j}) &
                    \vect{\tau}(\uvect{k})}
                    \cdot\uvect{n} \\    
    \vect{\tau}(\uvect{n}) &=
                \mat{\sigma_{ii} & \sigma_{ij} & \sigma_{ik}
                    \\
                    \sigma_{ji} & \sigma_{jj} & \sigma_{jk}
                    \\
                    \sigma_{ki} & \sigma_{kj} & \sigma_{kk}}
                    \cdot\uvect{n}
}
We call $\vect{\tau}$ the traction vector and $\tensor{\sigma}$ the stress tensor.
\eqn{
\tensor{\sigma}=\mat{\vect{\tau}(\uvect{i}) &
                    \vect{\tau}(\uvect{j}) &
                    \vect{\tau}(\uvect{k})
                    } =
                \mat{\sigma_{ii} & \sigma_{ij} & \sigma_{ik}
                    \\
                    \sigma_{ji} & \sigma_{jj} & \sigma_{jk}
                    \\
                    \sigma_{ki} & \sigma_{kj} & \sigma_{kk}}
}
\eqn{
    \vect{\tau}(\vect{x},t,\uvect{n}) = \tensor{\sigma}(\vect{x},t)\cdot\uvect{n}
}
\todo{We show that the stress tensor is symmetric.}
Since a fluid continuously deforms under shear stress, a static, non-deforming fluid element in consequently under no shear stress. Hence for an arbitrary surface element in a fluid at rest, stress is in the normal direction. From equilibrium arguments, it can be proven that stress in a fluid at rest is isotropic.
\begin{equation}
\begin{split}
    \vect{\tau} &= -p(\vect{x},t)\vect{n} \\
    \vect{\tau} &= -p(\vect{x},t)\tensor{\delta}\cdot\vect{n}
\end{split}
\end{equation}
%%%%%%%%%%%%%%%%%%%%%%%%%%%%%%%%%%%%%%%%%%%%%%%%%%%%%%%%%%%%%%%%%%%%%%%%%%%%%%%%

We still need to relate the stress tensor to the flow field. The \textit{Newtonian Model} is based on the following assumptions:
\begin{enumerate}
    \item shear stress is proportional to the rate of shear strain in a fluid particle;
    \item shear stress is zero when the rate of shear strain is zero;
    \item the stress to rate-of-strain relation is isotropic—that is, there is no preferred orientation in the fluid.
\end{enumerate}
From the first assumption, we have $\sigma_{ij}=\mathrm{K}_{ijkl}e_{kl}$ where $\mathrm{K}$ is a fourth order tensor.

\todo{derive}
\begin{equation}
\begin{split}
\tensor{\sigma}&= -p\tensor{\delta}+\tensor{\tau_v} \\
&= -p\tensor{\delta} + \mu\left(\del\vect{v}+\del\vect{v}^\transp - \frac{2}{3}(\del\cdot\vect{v})\tensor{\delta}\right)
\end{split}
\end{equation}
%%%%%%%%%%%%%%%%%%%%%%%%%%%%%%%%%%%%%%%%%%%%%%%%%%%%%%%%%%%%%%%%%%%%%%%%%%%%%%%%
%%%%%%%%%%%%%%%%%%%%%%%%%%%%%%%%%%%%%%%%%%%%%%%%%%%%%%%%%%%%%%%%%%%%%%%%%%%%%%%%
\section{Reynolds Transport Theorem and Consequences}
We begin with the derivation for the Reynolds Transport Theorem which relates time rate of change of quantities in material volumes to the distribution of said properties in the volume. Consider an arbitrary, finite (finite, nonzero measure) control volume $\Omega(t)\subset\mathbb{R}^3$ bounded by some control surface $\partial\Omega(t)\subset\mathbb{R}^3$ in some time varying flow.
For some property $\phi(\vect{x},t)$,
\eqn{
    \ddd{t}\int_{\Omega(t)}\phi(\vect{x},t)\d\vect{x} = \lim_{\Delta t\rightarrow 0}\dfrac{\int_{\Omega(t+\Delta t)}\phi(\vect{x},t+\Delta t)\d\vect{x} - \int_{\Omega(t)}\phi(\vect{x},t)\d\vect{x}}{\Delta t}
}
For reasonably smooth  $\phi(\vect{x},t),$ use the taylor expansion of $\phi(\vect{x},t)$ about $\phi(\vect{x},t)$
\eqn{
    \phi(\vect{x},t+\Delta t) = \phi(\vect{x},t)+\Delta t\p_{t}\phi(\vect{x},t) + \mathcal{O}(\Delta t^2)
}
Substituting,
\eqn{
    \ddd{t}\int_{\Omega(t)}\phi(\vect{x},t)\d\vect{x} &= \lim_{\Delta t\rightarrow 0}
    \int_{\Omega(t+\Delta t)}\p_{t}\phi(\vect{x},t)\d\vect{x} + 
    \dfrac{\int_{\Omega(t+\Delta t)}\phi(\vect{x},t)\d\vect{x} - \int_{\Omega(t)}\phi(\vect{x},t)\d\vect{x} + \mathcal{O}(\Delta t^2)}{\Delta t}
     \\
    &= \int_{\Omega(t)}\p_{t}\phi(\vect{x},t)\d\vect{x} + \lim_{\Delta t\rightarrow 0} + 
    \dfrac{\int_{\Omega(t+\Delta t)}\phi(\vect{x},t)\d\vect{x} - \int_{\Omega(t)}\phi(\vect{x},t)\d\vect{x}}{\Delta t}
}
To explain the thought process, without loss of generality, let $\phi=1$. We are looking for the difference in volume between the two integrals. That is equal to the net volume that the boundary has expanded into. For an infinitesimal patch $\d \vect{S_x}$ on $\partial\Omega$ (containing point $\vect{x}$, and with outward normal $\uvect{n}$), that is
\eqn{
    \left(\vect{x}(t+\Delta t)-\vect{x}(t)\right)\cdot\uvect{n}\d \vect{S_x}
}
Capturing the difference with a Taylor expansion and integrating over $\partial\Omega$, we get
\eqn{
    &\int_{\Omega(t+\Delta t)}\phi(\vect{x},t)\d\vect{x} - \int_{\Omega(t)}\phi(\vect{x},t)\d\vect{x} =\\
    &\int_{\partial\Omega(t)} \phi(\vect{x},t)\cdot(\Delta t\vect{v_c})\cdot\uvect{n}\d \vect{S_x} +
    \int_{\partial\Omega(t)} \phi(\vect{x},t)\cdot(\ddd{t}\vect{v}_c\dfrac{\Delta t^2}{2})\cdot\uvect{n}\d \vect{S_x} + \mathcal{O}(\Delta t^3)
}
where $\vect{v_c}(\vect{x},t)$ is the velocity of the control surface at time $t$ at point $\vect{x}\in\partial\Omega(t)$ and $\uvect{n}$ is the outward pointing unit normal vector on $\partial\Omega(t)$. An infinitesimal area element $\d \vect{S_x}\subset\partial\Omega(t)$ moves a distance of $(\Delta t\vect{v_c}\cdot\uvect{n}+\mathcal{O}(\Delta t^2))$ normal to the surface in time $\Delta t.$ Another way to think of this approximation is the following: project the value of $\phi(\vect{x},t)$ on $\d \vect{S_x}\subset\partial\Omega(t)$ throughout the volume $\d \vect{S_x}(\Delta t\vect{v_c}\cdot\uvect{n})+\mathcal{O}(\Delta t^2)$
\eqn{
    \ddd{t}\int_{\Omega(t)}\phi(\vect{x},t)\d\vect{x} 
    &= \int_{\Omega(t)}\p_{t}\phi(\vect{x},t)\d\vect{x} + \lim_{\Delta t\rightarrow 0}
    \dfrac{\int_{\partial\Omega(t)}\phi(\vect{x},t)(\Delta t\vect{v_c})\cdot\uvect{n}dA+\mathcal{O}(\Delta t^2)}{\Delta t} \\
    &= \int_{\Omega(t)}\p_{t}\phi(\vect{x},t)\d\vect{x} + \int_{\partial\Omega(t)}\phi(\vect{x},t)\vect{v_c}\cdot\uvect{n}dA
}
$\p_{t}\phi$ is the local the rate of production (or accumulation) of $\phi.$ It accounts for the effects of change in $\phi$ in the interior of the domain (say due to creation/dissipation or advection/diffusion). The divergence term accounts for the capture or loss of $\phi$ by the motion of the control surface. If $\Omega(t)$ is a material volume then, $\vect{v_c}=\vect{v},$ at all points in $\partial\Omega(t)$. Hence, the final form of the Reynolds Transport Theorem for material volumes $\Omega(t)$ is:
\eqn{
    \ddd{t}\int_{\Omega(t)}\phi(\vect{x},t)\d\vect{x} 
    &= \int_{\Omega(t)}\p_{t}\phi(\vect{x},t)\d\vect{x} + \int_{\partial\Omega(t)}\phi(\vect{x},t)\vect{v}\cdot\uvect{n}\d \vect{S_x}
}
We now apply the well known divergence theorem, also referred to as Gauss' Law,
\eqn{
    \int_{\partial\Omega}\vect{\psi}\cdot\uvect{n}\d\vect{S_x} =
    \int_\Omega\grad\cdot\vect{\psi}\d\vect{x}
}
to get
\eqn{
    \ddd{t}\int_{\Omega(t)}\phi(\vect{x},t)\d\vect{x} 
    &= \int_{\Omega(t)}\p_{t}\phi+\grad\cdot(\vect{v}\phi)\d\vect{x} \\
    &= \int_{\Omega(t)}\DDD\phi+\phi(\grad\cdot\vect{v})\d\vect{x} \\
}
Since a material volume always contains the same fluid elements, time rate of change of total mass in a material volume should be zero.
\begin{equation}
\begin{split}
    m(t) &= \int_{\Omega}\rho\d V \\
    0 &= \ddd{t}m = \int_{\Omega}\p_{t}\rho +
    \del\cdot(\rho\vect{v}) \d V
\end{split}
\end{equation}
Since the integral is zero for arbitrary material volumes, the integrand must be zero.
\begin{equation}
\begin{split}
    \p_{t}\rho + \del\cdot(\rho\vect{v}) &= 0 \\
    \D_t\rho = \p_{t}\rho + (\vect{v}\cdot\del)\rho &= -\rho(\del\cdot\vect{v})
\end{split}
\end{equation}

Let $\vect{p}(t)$ denote the total momentum of a fluid element.
\begin{equation}
\begin{split}
    \vect{p}(t) &= \int_{\Omega}\rho\vect{v}\d V \\
    \ddd{t}\vect{p} &= \int_{\Omega}\p_{t}(\rho\vect{v}) + \del\cdot(\vect{v}\otimes\rho\vect{v})   \d V \\
    &=\int_{\Omega}\p_{t}(\rho\vect{v}) + \rho\vect{v}(\del\cdot\vect{v}) + (\vect{v}\cdot\del)(\rho\vect{v}) \d V \\
    &=\int_{\Omega}\D_t(\rho\vect{v}) + \rho\vect{v}(\del\cdot\vect{v}) \d V
\end{split}
\end{equation}
Consider the total force applied on the fluid element.
\eqn{
\vect{F}_\text{ext} &= \int_{\Omega}\rho\vect{g}\d V + \int_{\partial\Omega}\tensor{\sigma}\cdot\vect{n}\d A \\
&= \int_{\Omega}\rho\vect{g} + \del\cdot\tensor{\sigma}\d V \\
&= \int_{\Omega}\rho\vect{g} + \del\cdot\Big(-p\tensor{\delta} + \mu(\del\vect{v}+\del\vect{v}^\transp) + \lambda(\del\cdot\vect{v})\tensor{\delta}\Big) \d V \\
&= \int_{\Omega}\rho\vect{g} - \del p + \mu(\del^2\vect{v} + \del(\del\cdot\vect{v})) + \lambda\del(\del\cdot\vect{v}) \d V \\
&= \int_{\Omega}\rho\vect{g} - \del p + \mu\del^2\vect{v} + (\mu+\lambda)\del(\del\cdot\vect{v}) \d V
}
From Newton's second law, time rate of change of momentum is equal to the sum of all external forces applied.
\eqn{
0 &= \ddd{t}\vect{p} - \vect{F}_\text{ext} \\
&= \int_{\Omega} \p_{t}(\rho\vect{v}) + \rho\vect{v}(\del\cdot\vect{v}) + (\vect{v}\cdot\del)(\rho\vect{v}) - \bigg(
\rho\vect{g} - \del p + \mu\del^2\vect{v} + (\mu+\lambda)\del(\del\cdot\vect{v})
\bigg)
\d V
}
Since the integral is zero for an arbitrary domain (fluid element) $\Omega(t),$ it follows that the integrand must be zero everywhere.
\eqn{
\p_{t}(\rho\vect{v}) + \rho\vect{v}(\del\cdot\vect{v}) + (\vect{v}\cdot\del)(\rho\vect{v}) &=
\rho\vect{g} - \del p + \mu\del^2\vect{v} + (\mu+\lambda)\del(\del\cdot\vect{v})
\\
\vect{v}\cancelto{0\,\mathrm{(continuity)}}{(\p_{t}\rho+\del\cdot\vect{v}+\vect{v}\cdot\grad\rho)} + \rho\D_t\vect{v}
&= \rho\vect{g} - \del p + \mu\del^2\vect{v} + (\mu+\lambda)\del(\del\cdot\vect{v})
\\
\rho(\p_{t}\vect{v}+(\vect{v}\cdot\grad)\vect{v}) &= \rho\vect{g} - \del p + \mu\del^2\vect{v} + (\mu+\lambda)\del(\del\cdot\vect{v})
}
We have arrived at the Navier-Stokes equations.
%-------------------------------------------------------------------------
Take the divergence of the momentum equation.
\eqn{
\rho(\p_{t}(\del\cdot\vect{v})+ \grad\vect{v}:\grad\vect{v}^\transp + (\vect{v}\cdot\grad)(\del\cdot\vect{v})) &= \rho(\del\cdot\vect{g}) - \del^2 p + \mu\del^2(\del\cdot\vect{v}) + (\mu+\lambda)\del^2(\del\cdot\vect{v})
\\
\rho\D_t(\del\cdot\vect{v}) &= \del^2(-p+(\lambda+2\mu)(\del\cdot\vect{v})) + \rho(\del\cdot\vect{g}-\grad\vect{v}:\grad\vect{v}^\transp)
}
where $\tensor{A}:\tensor{B}=A_{ij}B_{ji}$ is the double dot product. This equation simplifies into a laplace equation for pressure in case of incompressible flow.
\eqn{
-\del^2p = \rho(\grad\vect{v}:\grad\vect{v}^\transp - \del\cdot\vect{g})
}
%-------------------------------------------------------------------------
We take the curl of the momentum equation. Define $\vect{\omega}=\del\cross\vect{v}$
\eqn{
\del\cross\rho(\p_{t}\vect{v}+(\vect{v}\cdot\grad)\vect{v}) &= 
\del\cross(\rho\vect{g} - \del p + \mu\del^2\vect{v} + (\mu+\lambda)\del(\del\cdot\vect{v}))
\\
\grad\rho\cross\D_t\vect{v} + \rho\D_t\vect{\omega} + \epsilon_{ijk}v_{l,i}v_{j,l} &=
\del\cross\vect{g} + \grad\rho\cross\vect{g} + \mu\del^2\vect{\omega} + \epsilon_{ijk}(-p+(\del\cdot\vect{v}))_{,ji}
\\
\grad\rho\cross(\D_t\vect{v}-\vect{g}) + \rho\D_t\vect{\omega} + \epsilon_{ijk}v_{l,i}v_{j,l} &=
\del\cross\vect{g} + \mu\del^2\vect{\omega} + \epsilon_{ijk}(-p+(\del\cdot\vect{v}))_{,ji}
}


Informally, the Laplacian operator $\del^2$ gives the difference between the average value of a function in the neighborhood of a point, and its value at that point. Thus, if $u$ is the temperature, $\del^2 u$ tells whether (and by how much) the material surrounding each point is hotter or colder, on the average, than the material at that point.

We explore energy conservation properties of the incompressible Navier Stokes equations. Define the energy in a fluid element $\Omega(t)$ as $\mathcal{E}=\int_{\Omega(t)}\frac{1}{2}\rho\abs{\vect{v}}^2\d\vect{x}$. Then,
\eqn{
    \ddd{t}\mathcal{E}
    &=\ddd{t}\int_{\Omega(t)}\frac{1}{2}\rho v_jv_j\d\vect{x}
     =\int_{\Omega(t)}\frac{1}{2}\rho\D_t(v_jv_j)\d\vect{x}\\
    &=\int_{\Omega(t)}\rho \vect{v}^\transp\D_t(\vect{v})\d\vect{x}
     =\int_{\Omega(t)}\vect{v}^\transp\left(-\grad p + \mu\del^2\vect{v}\right)\d\vect{x}\\
    &=\int_{\Omega(t)}-\del\cdot(p\vect{v}) + \mu v_jv_{j,ii} \d\vect{x}\\
    &=\int_{\Omega(t)}-\del\cdot(p\vect{v}) + \mu\left( (v_jv_{j,i})_{,i} -v_{j,i}v_{j,i} \right)  \d\vect{x}\\
    &=\int_{\partial\Omega(t)} \vect{v}\cdot\left( -p\uvect{n}+ \mu\uvect{n}\cdot\grad\vect{v} \right) \d\vect{x} - \int_{\Omega(t)}\mu v_{j,i}v_{j,i}\d\vect{x}
}


%%%%%%%%%%%%%%%%%%%%%%%%%%%%%%%%%%%%%%%%%%%%%%%%%%%%%%%%%%%%%%%%%%%%%%%%%%%%%%%%
\section{Stokeslet and Stresslets}
We discuss properties of fundamental solution to the Stokes equation. But first, review of potential theory..?
\eqn{
    \label{eqn:stokes_steady}
    \grad p - \mu\del^2\vect{v} &= \vect{f}\cdot\delta(\vect{x})\\
    \grad\cdot\vect{v} &= 0
}
