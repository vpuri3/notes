\chapter{Analysis Repository}

%%%%%%%%%%%%%%%%%%%%%%%%%%%%%%%%%%%%%%%%%%%%%%%%%%%%%%%%%%%%%%%%%%%%%%%%%%%%%%%%%%%%%%%%%%%%%
\section{$\mathbb{R}$}
Relations, Sets, and Functions

\begin{definition}[Relation]
    A relation $\sim$ (or $R$) on nonempty set $X$ is a subset of the set $X\cross X$. For $a,\,b\in X$, if $a\sim b$, then $(a,b)\in R$.
\end{definition}

Properties of relations: $\forall a,\,b\,c\in X,$
\begin{enumerate}
    \item Reflexive: $a\sim a$
    \item Symmetric: $a\sim b \iff b\sim a$
    \item Transitive:
        \eqn{
            \left.
            \begin{split}
                a\sim b\\
                b\sim c
            \end{split}
            \right\}\implies a\sim c
        }
    \item Antisymmetric:
        \eqn{
            \left.
            \begin{split}
                a\sim b\\
                b\sim a
            \end{split}
            \right\}\implies a = b
        }
\end{enumerate}

A relation that is reflexive, symmetric, and transitive is called equivalent relation. For $a\in X,$ the equivalent class of $a$ is
\eqn{
    \left[a\right]\defeq\left\{ x\in X | a\sim x \right\}\hspace{1em}
}
Equivalent classes of elements in $X$ form a partition of $X.$ An ordered relation is a relation that is reflexive, transitive, and antisymmetric, usually denoted by $\leq.$

\todo{functions (one-one, onto), fields, ordered fields, vector spaces}

\begin{theorem}[Pigeonhole Principle]
    If $n$ items are put into $m$ containers, with $n>m$, then there exists at least one container with more than one item.
\end{theorem}

\begin{theorem}
    For $a\in\mathbb{R}\setminus\mathbb{Q}$, the set $[na]=\left\{[na]|\n\in\mathbb{N}\right\}$ of fractional parts is dense in $[0,1]$.
\end{theorem}
\begin{proof}
    Fix $\epilon>0$. In this proof, $[\cdot]$ denotes the the fractional part of the argument. We aim to show that for every $x\in[0,1]\exists n\st \abs{x-[na]}<\epsilon$. Starting with $x=0,$ fix $M\st\frac{1}{M}<\epsilon.$ If $[0,1]$ is partitioned into $M$ panels of size $\frac{1}{M}$, then, by the Pigeonhole principle, $\exists n_1,\,n_2\in\{1,\dotsc,M+1\}\st$ $[n_1a],\,[n_2a]$ lie in the same panel. $\implies [(n_2-n_1)a]<\frac{1}{M}<\epsilon.$ Hence $0$ is a limit point of $[na]$.
    
    Now consider $x\in(0,1]$. Find $n\st[na]<\frac{1}{M}<\epsilon$. If $x\in[0,\frac{1}{M}],$ we are done. Else, let $x$ belong to the $j^\mathrm{th}$ panel: $x\in[\frac{j}{M},\frac{j+1}{M}]$. Since $[na]<\frac{1}{M},\exists p\in\mathbb{N}\st p[na]$ belongs to the $(j-1)^\mathrm{th}$ panel. Setting $p=\sup\{s\in\mathbb{N}|s[na]<\frac{j}{M}\}$. $\implies \abs{x - [(p+1)na]}<\frac{1}{M}<\epsilon.$
\end{proof}

\todo{$\R,$ the completion of $\Q$}\\
\todo{Construction of $\R$ using Dedekind's Cuts}

\begin{theorem}[Archimedean Property]
   $\forall x,y\in\mathbb{R}^+\,\exists N\in\mathbb{N}:y<nx$ 
\end{theorem}
\begin{proof}
   Fix $x,\,y\in\mathbb{R}$. Let $A=\mathbb{N}x$. Consider, \textit{ad absurdum,} $\forall n\in\mathbb{N},\,y\geq nx.$ That is, $y$ is an upper bound for $A.$ Let $\al=\sup{A}=n\al$ for some $n\in\mathbb{N}.$ As $x>0,$ $\al<\al+x=(n+1)x\in A$. Therefore $\al$ is not an upper bound for $A.$
\end{proof}

\begin{theorem}
   $\mathbb{Q}$ is dense in $\mathbb{R},$ i.e. between any two reals, there exists a rational number.
\end{theorem}
\begin{proof}
   Let $x,y\in\mathbb{R},x<y.$ Without loss of generality, consider the case where $x,y>0.$ Applying the Archimedean property, we get
   \eqn{
        &\exists n\in\mathbb{N}, n(y-x)>1\\
        &\implies x < x+\frac{1}{n} < y\\
        &\implies x < \frac{nx+1}{n} < y
   }
   Since the numerator is not guaranteed to be rational, we seek $m\in\mathbb{N}$ such that $nx<m<nx+1$. Apply the Archimedean property again to find positive integer $m_1$ such that $0<nx<m_1.$ Hence there exists $0<m\leq m_1$ such that
   \eqn{
       &m<nx<m+1\\
       &nx<m<nx+1<ny\\
       &x<\frac{m}{n}<y
   }
\end{proof}

\begin{definition}[Dense subset]
    A set $S\subset X$ is dense in $X$ if every element in $X$ is a limit point of $S.$
\end{definition}

The real number system is an ordered field $\mathbb{R}$, a complete extension of $\mathbb{Q}$ that has the least upper bound property. 

\begin{definition}[Least Upper Bound Property]
    For $S\subset \mathbb{R}$, an upper bound of $S$ in $\mathbb{R}$ is an element $x\in\mathbb{R}$ such that $\forall s\in S, s<x.$ The least upper bound of $S$ in $\mathbb{R}$ is an element $y$ such that
    \begin{enumerate}
        \item $y$ is an upper bound for $S$
        \item if $x$ is an upper bound for $S,$ then $y<x.$
    \end{enumerate}
    The least upper bound property of $\mathbb{R}$ is that any nonempty set of real numbers that is bounded from above has a least upper bound.
\end{definition}

An ordered set satisfies the completeness axiom if every subset $S$ that is bounded above has a least upper bound denoted $\text{sup}S\in\mathbb{R}$. If $-S=\{-s|s\in S\},$ then $\text{inf}(S)=-\text{sup}(-S)$

\begin{theorem}[Knaster-Tarski Fixed Point Theorem]
    Consider a set $X\subset\mathbb{R}$, for which $a=\inf(X),\,b=\sup(X)\in X,$ then every increasing function $f:X\rightarrow X$ has at least one fixed point, i.e. $x_0\in X\st f(x_0)=x_0.$
\end{theorem}
\begin{proof}
    Consider the set $S=\left\{ x\in X | x\leq f(x) \right\}$. $S$ is nonempty since $a\in S.$ Hence $\beta=\sup(S)\in X$ must exist.
    \eqn{
                &\forall x\in S, x\leq \beta&\qquad&                          \\
        \implies& x\leq f(x),\,f(x)\leq f(\beta)   &&\text{($f$ increasing function)}\\
        \implies& \forall x\in S,\,x\leq f(\beta)\label{eqn:forall}\\
        \implies& \beta \leq f(\beta)              &&\text{(since $\beta=\sup(S)$ )}\\
        \implies& f(\beta)\leq f\left(f(\beta)\right)\\
    }

\end{proof}

We define the following for $x,y\in\mathbb{R}$
\eqn{
    \abs{x} &=\left\{
        \begin{split}
             x,\hspace{1em} x>0\\
            -x,\hspace{1em} x>0\\
        \end{split}
        \right.\\
    \min(x) &= \frac{x+y-\abs{x-y}}{2}\\
    \max(x) &= \frac{x+y+\abs{x-y}}{2}\\
}

\begin{definition}[Extended Real Line]
We define $\infty\defeq\sup\mathbb{R}$ and $-\infty\defeq\inf\mathbb{R}$, and call $\mathbb{R}\cup\left\{ -\infty,\infty\right\}$ the extended real line.
\end{definition}

\begin{theorem}[Existence of $n^\mathrm{th}$ root]
    $\forall a\in\mathbb{R}^+,\exists b\in\mathbb{R}+\st b^n=a$
\end{theorem}
\begin{proof}
    \underline{Uniqueness:} Say $b^n=c^n=a.$
    \eqn{
        \implies 0 &= b^n-a^n = (b-c)\left(b^{n-1}+ ab^{n-2}+\dotsb+a^{n-1}\right)\\
    }
    Since $a,\,b,\,c>0,$ $b-c=0$.
    \underline{Existence:} Consider the function $f:[0,a+1]\to\mathbb{R}$
    \eqn{
        f(x) &= x + \frac{a-x^n}{n(a+1)^{n-1}}\\
    }
    Observe that $f(x)$ is increasing, and that all fixed points of $f$ satisfy $x=a^n$. To prove that $f$ is increasing, let $0<x\leq y\leq a+1$
    \eqn{
        f(y)-f(x) &= (y-x) - \frac{y^n-x^n}{n(a+1)^{n-1}} \\
        &= (y-x)\left(1 - \frac{\sum_{k=0}^{n-1}x^{k}y^{n-1-k}}{n(a+1)^{n-1}}\right)\\
        &\geq 0\\
    }
    Now we eliminate $x=0,\,a+1$ as fixed points, and apply the fixed point theorem.
    \eqn{
        0 < f(0) &= \frac{a}{n(a+1)^{n-1}} < f(a+1) = \frac{(n-1)(a+1)^n+a}{n(a+1)^{n-1}} < \frac{(n-1)(a+1)^n+(a+1)^n}{n(a+1)^{n-1}} = a+1
    }
\end{proof}

%%%%%%%%%%%%%%%%%%%%%%%%%%%%%%%%%%%%%%%%%%%%%%%%%%%%%%%%%%%%%%%%%%%%%%%%%%%%%%%%%%%%%%%%%%%%%
\section{Sequences}
\begin{definition}[Sequence]
    A sequence $(a_n)_{n\in\mathbb{N}}$ of elements in a set $X$ is a function $a_\cdot:\mathbb{N}\rightarrow X$.
\end{definition}

\begin{definition}[Limit of a sequence]
    A sequence $(x_n)$ converges to $x\in X$ in norm $\norm{\cdot}$, abbreviated as $x_n\xrightarrow{n} x$, if
    \eqn{
        \forall\epsilon>0,\exists N\in\mathbb{N} \st \forall n\geq N, \norm{x-x_n}<\epsilon
    }
    The element $\lim_{n\rightarrow\infty}x_n=x$ is called the limit of sequence $x_n.$
\end{definition}


We will consider sequences of real numbers under the absolute value norm. Below listed are some properties of convergent sequences.

\begin{enumerate}
    \item The limit of a convergent sequence is unique.
    \begin{proof}
        $\epsilon>0$. Let $x_n\rightarrow a,\,x_n\rightarrow b\in\mathbb{R}$. Without loss of generality, let $b>a,$ and fix $\epsilon=\dfrac{b-a}{3}$. Then $\exists N_a,N_b\in\mathbb{N}\st$
        \eqn{
        &\forall n>N_a, \abs{a-x_n}<\epsilon\\
        &\forall n>N_b, \abs{b-x_n}<\epsilon\\
        }
        Then, $\forall n>\max(N_a,N_b),$
        \eqn{
        \implies&\abs{b-a} < \abs{a-x_n}+\abs{b-x_n} < (b-a)\frac{2}{3} \contradiction
        }
    \end{proof}    
    
    \item $(a_n)$ convergent $\implies \{a_n\}_n$ bounded.
    \begin{proof}
        Let $a_n\rightarrow a.$ $\forall \epsilon, \exists N\st \forall n>N,\abs{a-a_n}<\epsilon.$ Therefore,
        \eqn{
            \forall n,\abs{a_n} < \max\big( \{\abs{a_m}\}_{m=1}^{N-1}\cup\{\abs{a\pm\epsilon}\}\big)
        }
    \end{proof}
    
    \item 
    \eqn{
        \begin{array}{r}
        a_n\rightarrow a \\
        b_n\rightarrow b
        \end{array}\bigg\}
        \implies a_nb_n\rightarrow ab
    }
    \begin{proof}
       Fix $\epsilon>0$. For $\epsilon>0,$ $\exists N_a,\,N_b\st$
       \eqn{
           &\forall n>N_a \abs{a-a_n}<\epsilon_a>0\\
           &\forall n>N_b \abs{b-b_n}<\epsilon_b>0\\
       }
       Since $b_n$ is a convergent sequence, $\exists \abs{b_n} < M_b\in\mathbb{R}$. Let $\epsilon_a=\dfrac{\epsilon}{2M_b},\,\epsilon_b=\dfrac{\epsilon}{2a}$. For $n>\max(N_a,N_b),$
       \eqn{
            \abs{ab-a_nb_n} &= \abs{ab+ab_n-ab_n-a_nb_n}\\
            &\leq a\abs{b-b_n}+b_n\abs{a-a_n}\\
            &< a\frac{\epsilon}{2a} + b_n\frac{\epsilon}{2M_b}\\
            &< \epsilon
       }
    \end{proof}
    
    \item $a_n\rightarrow a\neq0\implies \dfrac{1}{a_n}\rightarrow\dfrac{1}{b}.$
    \begin{proof}
        Fix $0<\epsilon<\frac{\abs{b}}{2}$ (bounding $b_n$ away from $0$). $\exists N\st\forall n>N \abs{b-b_n}<\epsilon_b.$
        \eqn{
            \implies\abs{\frac{1}{b}-\frac{1}{b_n}} &= \abs{\frac{b-b_n}{bb_n}}\\
            &< \frac{\epsilon_b}{\abs{b}\abs{b_n}}\\
            &< \frac{2\epsilon_b}{\abs{b}^2}
        }
        Allowing $\epsilon_b=\frac{\abs{b}^2\epsilon}{2}$, we complete the proof.
    \end{proof}
    
    \item 
    \begin{theorem}[Squeeze Theorem]
        $\forall n>N,L\leftarrow a_n \leq b_n \leq c_n\rightarrow L\in\mathbb{R}\implies b_n\rightarrow L.$
    \end{theorem}
    \begin{proof}
        $\forall n,\abs{b_n-a_n}<c_n-a_n\rightarrow L-L=0$. Fix $\epsilon>0$. $\exists N>0\st\forall n>N,\abs{L-a_n}<\dfrac{\epsilon}{2},\,\abs{c_n-a_n}<\dfrac{\epsilon}{2}$
        \eqn{
            \implies \abs{L-b_n} &= \abs{L-a_n+a_n-b_n}\\
            &< \abs{b_n-a_n} + \abs{a_n-L}\\
            &< \frac{\epsilon}{2} + \frac{\epsilon}{2}\\
            &= \epsilon
        }
    \end{proof}
    
    \item
    \eqn{
        \begin{array}{r}
             a_n\rightarrow a  \\
             \forall n, a_n > 0
        \end{array}\bigg\}
        \implies \forall k\geq 1, \sqrt[k]{a_n}\rightarrow\sqrt[k]{a}
    }
    \begin{proof}
        Fix $0<\epsilon\leq\frac{\abs{a}}{2}$. When $a=0,\exists N\st\forall n>N \abs{a_n}<\epsilon^k.\implies \abs{\sqrt[k]{a_n}}<\sqrt[k]{\epsilon^k}=\epsilon.$ Consider when $a>0.$ Let $\epsilon_a>0.$ $\exists N\st\forall n>N,\abs{a-a_n}<\epsilon_a$
        
        We use the identity
        \eqn{
            A-B = \frac{A^k-B^k}{\sum_{i=0}^{k-1} A^iB^{k-1-i}}
        }
        When $A,\,B>0,$ we trivially have $\abs{A-B}<\dfrac{\abs{A^k-B^k}}{B^{k-1}}$
        
        \eqn{
            \forall n>N, 0 \leq \abs{\sqrt[k]{a}-\sqrt[k]{a_n}} &< \frac{\abs{a-a_n}}{\sqrt[k]{a^{k-1}}}
            < \frac{\epsilon_a}{\sqrt[k]{a^{k-1}}}
        }
        Allowing $\epsilon_a=\epsilon\sqrt[k]{a^{k-1}}$ completes the proof.
    \end{proof}
    
    \item $\abs{a_n}\rightarrow 0\implies a_n\rightarrow 0$
    \begin{proof}
        Apply squeeze theorem to $0\leftarrow(-\abs{a_n})\leq a_n \leq \abs{a_n}\rightarrow 0.$
    \end{proof}
    
    \item
    \begin{theorem}[Monotone Convergence Theorem]
    For $a_n$ bounded and monotone, $a_n\nearrow\sup\{a_n\}_n$ or $a_n\searrow\inf\{a_n\}$. (Note on notation: $a_n\nearrow L \iff a_n \text{ is a monotone increasing sequence }, a_n\rightarrow L.$)
    \end{theorem}
    \begin{proof}
        Consider the case when $a_n$ is a monotone increasing sequence. The assumption is without loss of generality, for if $a_n$ is monotone decreasing, we consider $-a_n$ and prove its convergence to $\sup\{-a_n\}_n=-\inf\{a_n\}_n.$ By the least upper bound property, we have a unique $L\defeq\sup\{a_n\}_n.$ We aim to prove that $\forall\epsilon>0\exists N\st\forall n>N,\abs{L-a_n}<\epsilon.$ If the statement is not true, then $\exists\epsilon_0>0\st\forall k\exists n_k>k,\abs{L-a_{n_k}}\geq\epsilon_0.$ Since $a_k\nearrow,$ we have $\forall k, a_k\leq a_{n_k}\leq L-\epsilon_0\implies (L-\epsilon_0)<L$ is an upper bound for $\{a_n\}.\contradiction$
    \end{proof}
    
\end{enumerate}

\begin{example}
    $-1<a<1\implies a^n\rightarrow0.$
    
    The example is trivial when $a=0$. Consider the case when $0<\abs{a}<1.$ We consider the case when $0<\abs{a}<1\implies\frac{1}{\abs{a}}> 1$. Let $\frac{1}{\abs{a}}\defeq 1+h.$
    \eqn{
        &\left(\frac{1}{\abs{a}}\right)^n = \left( 1+h\right)^n = 1+ nh + \mathcal{O}(h^2) > 1+nh\\
        &\implies\abs{a}^n < \frac{1}{1+nh}\\
        &\implies \abs{a}^n\rightarrow 0
    }
    Since $a^n \leq \abs{a^n} \leq \abs{a}^n$ we conclude that $a^n\rightarrow 0.$
\end{example}

\begin{example}
    $a_n=\sqrt[n]{n}$.
    
    For $n>1\implies \sqrt[n]{n}>\sqrt[n]{1}=1$. For $h_n>0$, let $a_n=1+h_n.$ We aim to show that $h_n\to0$.
    \eqn{
        \implies n &= a_n^n = (1+h_n)^n = 1 + nh_n + \frac{n(n-1)}{2}h_n^2++\frac{n(n-1)(n-2)}{3!}h_n^3\dotsb\\
        &\geq 1 + \frac{n(n-1)}{2}h_n^2
    }
    We get $h_n^2 \leq \frac{2}{n}\implies h_n\to0.$
\end{example}

\begin{example}
    $a_n=\dfrac{a^n}{n!}$ for fixed $a>0$.
    
    We consider the ration $\dfrac{a_{n+1}}{a_n} = \dfrac{a}{n+1}<1$ for large enough $n$.
    \eqn{
        \left.\begin{split}
            &(a_n)_{n>n_0}\searrow\\
            &\forall n, a_n > 0\\
        \end{split}\right\}\implies a_n\to L \defeq\inf(a_n) \geq 0
    }
    If $L>0,\,\dfrac{a_{n+1}}{a_n}\xrightarrow{n}\dfrac{L}{L}=1=0\xleftarrow{n}\dfrac{a}{n+1}\contradiction$. Therefore, $L=0.$
\end{example}

\begin{example}[Decimal Expansion]
    Given $d_0\in\mathbb{Z},(d_n)_{n>0}\in\left\{0,1,2,\dotsc,9\right\}$, consider
    \eqn{
        s_n &= \sum_{i=0}^n \frac{d_i}{10^i}
    }
    The sequence is monotonically increasing, and
    \eqn{
        d_0 &\leq s_n \leq s_\infty = d_0 + \sum_{i=1}^\infty\frac{d_i}{10^i} \leq d_0 + 9\sum_{i=1}^\infty\frac{1}{10^i} \leq d_0 + 9\left(\frac{1}{1-\frac{1}{10}}-1\right) = d_0 + 1
    }
    \eqn{
        \left.
        \begin{split}
            d_0 \leq s_n \leq d_0+1 \\
            s_n \nearrow \\
        \end{split}
        \right\}\implies s_n \mathrm{convergent}
    }
    Hence, $s\defeq\lim_n s_n = d_0.d_1d_2\dotso\in\left[d_0,d_0+1\right]$. This representation is not unique as
    \eqn{
        0.99\dotso = \sum_{i=1}^\infty\frac{9}{10^i} = 9\left(\frac{1}{1-\frac{1}{10}}-1\right) = 1.00\dotso
    }
    We can recapture the digits in $s$ as follows: $d_0=\floor{s}$. Define $T(x)=\floor{10(s-d_0)}$. Now, $d_1 = T^1(s),\, d_n = T^n(x)$
\end{example}

\begin{definition}[Cauchy Sequence]
    \eqn{
        a_n \text{Cauchy sequence} \iff\forall\epsilon>0\exists N\st\forall m,\,n>N,\abs{a_m-a_n}<\epsilon
    }
\end{definition}
The elements in the tail of a cauchy sequence get arbitrarily close to each other. $\forall k,\abs{a_n - a_{n+k}}\xrightarrow{n}0$

\begin{proposition}
    $a_n$ convergent $\implies$ $a_n$ cauchy
\end{proposition}
\begin{proof}
    Suppose $a_n\to L$. Fix $\epilon > 0$, and sufficiently large $m,n$, we get $a_m,a_n\in B(L,\frac{\epsilon}{2})\implies$ maximum distance between $a_m,a_n$ is equal to the diameter of the ball. $\implies \abs{a_m-a_n}<\epilon$.
\end{proof}

\begin{example}[Harmonic Series]
    $s_n = 1 + \frac{1}{2} +\dotsb+\frac{1}{n}$.
    
    Clearly $s_{n+k}-s_n = \sum_{i=1}^k\frac{1}{n+i}$.
    \eqn{
        \implies s_{2n} - s_n &= \sum_{i=1}^n\frac{1}{n+i} \geq n \frac{1}{2n} = \frac{1}{2}
    }
    Since $s_n$ is not a cauchy sequence, it is not convergent either. And since, $s_n\nearrow$, $s_n$ diverges to infinity.
\end{example}

\begin{example}
    $s_n = 1 + \frac{1}{n^2} + \dotsb + \frac{1}{n^2}$
    
    Clearly, $s_n\nearrow.$ We try to find an upper bound for $s_n$
    \eqn{
        s_n &< 1 + \sum_{i=1}^n\frac{1}{i(i-1)} = 1 + \sum_{i=1}^n\frac{1}{i-1} - \frac{1}{i}\\
        &= 1 + \left(\frac{1}{1}-\frac{1}{2}\right) + \left(\frac{1}{2}-\frac{1}{3}\right) + \left(\frac{1}{3}-\frac{1}{4}\right) + \dotsb + \left(\frac{1}{n-2}-\frac{1}{n-1}\right) + \left(\frac{1}{n-1}-\frac{1}{n}\right) 
        &= 2 - \frac{1}{n}
    }
    Hence, $s_n$ convergent, and therefore cauchy.
\end{example}

\begin{example}
    $a_n = \rt{n}{n!}$
    
    We aim to show that the tail of $a_n$ is larger than any arbitrarily chosen real. To do so, fix integer $M>1$. Now, $\forall n>2M,$
    \eqn{
        n! &= 1\cdot2\dotsb M\cdot(M+1)\dotsb(2M) \\
        a_n &\geq M^{\frac{n-M}{n}} = M\frac{1}{\rt{n}{M^M}}
    }
    As $\rt{n}{M^M}\to1,\exists N>2M\st\forall n>N\rt{n}{M^M}>\frac{1}{2}$. Therefore, $\forall n>N,a_n>\dfrac{M}{2}$.
\end{example}

\begin{example}
    $s_n=s_n(x)=1+\dfrac{x}{1}+\dfrac{x^2}{2}+\dotsb\dfrac{x^n}{n!}$
    
    \eqn{
        s_{n+k}-s_n &= \sum_{i=1}^k\frac{x^{n+i}}{(n+i)!}
        = \frac{x^{n+1}}{(n+1)!}\left( 1 + \frac{x}{n+2} + \frac{x^2}{(n+2)(n+3)}+\dotsb+\frac{x^{k-1}}{(n+2)\dotsb(n+k)} \right) \\
        \abs{s_{n+k}-s_n} &\leq \frac{\abs{x}^{n+1}}{(n+1)!} \left( 1 + \frac{\abs{x}}{n+2} + \frac{\abs{x}^2}{(n+2)^2}+\dotsb+\frac{\abs{x}^{k-1}}{(n+2)^{k-1}} \right) \\
        &\leq \frac{\abs{x}^{n+1}}{(n+1)!}\sum_{j=0}^\infty\left(\frac{\abs{x}}{n+2}\right)^j\\
        &= \frac{\abs{x}^{n+1}}{(n+1)!}\frac{1}{1-\frac{\abs{x}}{n+2}}
        \\
    }
    We look at the upper bound in the limit $n\to\infty.$
    \eqn{
        \lim_n \frac{\abs{x}^{n+1}}{(n+1)!}\frac{1}{1-\frac{\abs{x}}{n+2}} &=
        \lim_n\frac{\abs{x}^{n+1}}{(n+1)!}\lim_n\frac{1}{1-\frac{\abs{x}}{n+2}} = 0\cdot1 = 0
    }
    Hence $s_n$ is cauchy and therefore convergent.
\end{example}

\begin{theorem}
    Cauchy $\implies$ bounded.
\end{theorem}
\begin{proof}
    Fix $\epsilon>0$. For cauchy sequence $a_n,\exists N\st\forall n>N,\forall k,\abs{a_{n+k}-a_n}<\epsilon$. The set $\{a_n\}_{n\in\mathbb{N}}$ lies in the interval $[-M,M]$ where
    \eqn{
        M \defeq \max\left(\left\{\abs{a_n}\right\}_{n<N} \cup \left\{\abs{a_N}+\epsilon\right\}\right)
    }
\end{proof}

\begin{definition}[Limit Inferior and Limit Superior]
Let $a_n$ be a sequence.
\eqn{
    \liminf_{x\rightarrow0}&=\lim_{n\to\infty}\sup_{k>n}a_k\\
    \limsup_{x\rightarrow0}&=\lim_{n\to\infty}\inf_{k>n}a_k\\
}
Limit inferior, and limit superior, are the limits of the infimum/supremum of the tail of the sequence $(a_k)_{k>n}$ in the limit $n\to\infty$. For every $n$, it is always true that a finite number of elements lie in the discarded part of the sequence, $(k<n)$, and an infinite number of elements lie in the tail, $(k>n)$, and $\liminf,\,\limsup$ are the bounds of the infinitely long tail. Consider $,\forall k,$ the truncated sequence $(a_n)_{n\geq k}$ and the set $A_k\defeq\left\{a_n\right\}_{n\geq k}$. Define $u_k=\inf A_k,\,v_k=\sup A_k$. Because
\eqn{
    A_1\supset A_2\supset A_3\dotsb A_{k-1}\supset A_k\\
}
we have,
\eqn{
    u_1\leq u_2\dotsb u_{k-1}\leq u_k\leq \dotsb \leq v_k \leq v_{k+1} \dotsb\leq v_2\leq v_1
}
If $(a_n)$ is bounded, $-\infty<u_1\leq v_1<\infty\implies u_n,v_n$ convergent (by monotone convergence theorem). Therefore,
\eqn{
    \liminf_n a_n &= \lim_k u_k = \sup_k u_k\\
    \limsup_n a_n &= \lim_k v_k = \inf_k v_k\\
}
Another way to look at $\liminf,\,\limsup$ is the following: $L_i=\liminf_n a_n$ if
\begin{enumerate}
    \item $\forall x<L_i,$ there exist infinite number of $a_n$ greater than $x$
    \item $\forall x>L_i,$ there exist infinite number of $a_n$ less than $x$
\end{enumerate}
\end{definition}

Some properties of $\liminf,\,\limsup$ are as follows:
\begin{enumerate}
    \item $\limsup_n a_n = -\liminf_n(-a_n)$
    \item $\limsup_n (a_n+b_n) \leq \limsup_n a_n + \limsup_n b_n$ (follows from $\sup(A + B)\leq \sup A + \sup B$)
\end{enumerate}

\begin{lemma}[Nested Interval Property]
    Let $I_n=\left[a_n,b_n\right]\subset \mathbb{R}$ be a sequence of intervals such that $I_n$ are nested as follows:
    \eqn{
        I_1 \supset I_2 \supset\dotsb \supset I_n \supset \dotsb
    }
    Then, $\cap_n I_n \neq \varnothing$. Further, if length of the interval is going to zero with $n,$ i.e., $(b_n-a_n)\to 0$, then $\cap_n I_n = \{x_0\}$ where $a_n,b_n\to x_0$.
\end{lemma}
\begin{proof}
    \todo{}
\end{proof}

\begin{definition}[Subsequence]
    A subsequence of a sequence $a_n$ is the sequence $(b_k)_k=(a_{n_k})_k$ where $n_\cdot:\mathbb{N}\rightarrow\mathbb{N}$ with the restriction that $\forall k, k\leq n_k.$
\end{definition}

\begin{definition}[Limit Point]
    (Also called accumulation point) A limit point $a_n$ is a limit of convergence of a subsequence $(a_{n_k})_k$.
\end{definition}

\begin{theorem}[Bolzano-Weierstrass]
\end{theorem}

\begin{lemma}[Finite intersection property]
\end{lemma}

%%%%%%%%%%%%%%%%%%%%%%%%%%%%%%%%%%%%%%%%%%%%%%%%%%%%%%%%%%%%%%%%%%%%%%%%%%%%%%%%
%\section{Series}
\todo{those silly tests}

\begin{theorem}[Rearrangement theorem]
\end{theorem}

\begin{theorem}[Fubini-Tonelli]
Summability of infinite matrices
\end{theorem}

\begin{theorem}[Summability of infinite matrices]
\end{theorem}

%%%%%%%%%%%%%%%%%%%%%%%%%%%%%%%%%%%%%%%%%%%%%%%%%%%%%%%%%%%%%%%%%%%%%%%%%%%%%%%%
\section{Sets, Spaces and Functions}

For now, we restrict ourselves to sets in finite dimensional spaces.

Norm

\begin{theorem}[Cauchy-Schwarz inequality]
\end{theorem}

Open, closed/ closure, interior, etc.

DeMorgan's Laws (Unions and Intersections)

The Cantor set

\begin{definition}[Open Ball]
    The open ball of radius $r$ around $x_0\in X$ is defined as follows:
    \eqn{
        B(x_0,r) = \{ x\in X | \norm{x-x_0} < r \} \subset X
    }
\end{definition}

\begin{definition}[Continuity]
    We say a function $f:X\rightarrow Y$ is continuous at $x_0\in X$
\end{definition}


\begin{definition}[Compact Set]
    A set for which every open cover contains a finite subcover.
\end{definition}
\begin{theorem}
    Compact $\iff$ sequentially compact
\end{theorem}
\begin{proof}
    The spirit of compactness is ``any bounded sequence has a sequence converging in the set.''
\end{proof}

\begin{definition}[Precompact Set]
    A set in a normed vector space $X$ is precompact (also called relatively compact set) if its closure is compact.
\end{definition}

equivalance between different definitions of continuity.

\todo{
darboux integral, shuffling limits of integrals, prove everything from calculus
}

\begin{definition}[Fixed Point Theorem]
    A \textbf{fixed point} of $T:X\rightarrow X$ is an element $x_0\in X\st T(x_0)=x_0$. A fixed point is \textbf{globally attracting} if $\forall x\in X, T^n(x)\xrightarrow{n}x_0$
\end{definition}
We call $T$ a \textbf{contraction} if $\forall x_1,x_2\in X, d(Tx_1,Tx_2)\leq \al d(x_1,x_2),\al\in[0,1)$
\begin{theorem}[Contraction Mapping Principle]
    For a contraction map $T:X\rightarrow X$ on a complete metric space $X$, there exists a unique fixed point $x_0\in X$. The fixed point is globally attracting with geometric convergence rate.
\end{theorem}
\begin{proof}
    Uniqueness: Suppose $x_1,x_2\in X\st x_1=Tx_1,\,x_2=Tx_2$. Then $T$ is no longer a contraction since $d(Tx_1,Tx_2)=d(x_1,x_2)$.
    
    $\forall x\in X$, we show that the sequence $T^nx$ is cauchy due to the contraction property.
    \eqn{
        d(T^{n+1}x,T^nx) \leq \al d(T^nx,T^{n-1}x)\leq\dotsb \leq \al^nd(Tx,x)
    }
    This implies, the distance between nonconsecutive iterates is bounded by
    \eqn{
        d(T^{n+m}x,T^nx) &\leq d(T^{n+m}x,T^{n+m-1}x)+\dotsb+d(T^{n+1}x,T^nx)\\
        &\leq(\al^{n+m-1}+\dotsb+\al^{n})d(Tx,x)\\
        &\leq\al^n\frac{1-\al^m}{1-\al}d(Tx,x)\\
        &\leq\frac{\al^n}{1-\al}d(Tx,x)\\
    }
    Let $x_0=\lim_n T^nx$ for some $x$. $\implies Tx_0=T\lim_n T^nx = \lim_n T^{n+1}x= x_0$.
\end{proof}

\begin{theorem}[Contraction Power Mapping Principle]
    For $T:X\rightarrow X$, if $T^m$ is a contraction, then $T$ has a fixed point that is globally attracting.
\end{theorem}
\begin{proof}
    Uniqueness is clear since any fixed point of $T$ is a fixed point of $T^m$, and $T^m$ has a unique fixed point that is globally attracting. To prove existence, we observe that
    \eqn{
        &\forall x, T^{mn}x \xrightarrow{n} x_0\\
        \implies &T^{mn+k}x \xrightarrow{n} x_0\\
        \implies &d\left(T^{mn+k}x,T^{mn}x\right) \leq \al^n d\left(T^kx,x\right)\rightarrow0
    }
    illustrating that for sufficiently large $N=mn,\,d\left(T^{N+k}x,T^Nx\right)\rightarrow0$.
\end{proof}

Application to picard iteration, fredhold and volterra integral equations are straightforward. Existence and uniqueness of solutions are obtained from fixed point theorems by establishing that the operators are contractions (provided we are working in Banach spaces). However contractivity is a rather strong property. We now prove the existence of a fixed point assuming compactness of a mapping.

\begin{definition}[Compact Operator]
    An operator $T:X\rightarrow Y$ over normed vector spaces is compact if it is continuous, and for every bounded sequence $(x_n)\in X$, $Tx_n$ has a convergent subsequence. In other words, every bounded set is mapped to a precompact set.
\end{definition}